\begin{prim}
Посчитать скалярное произведение, если
$\textbf{x}=\left(\begin{array}{r}
      -1 \\
      -1 \\
      1 
\end{array}\right), \textbf{y}=\left(\begin{array}{r}
      0\\
      1 \\
      3
\end{array}\right), \Gamma = \left(\begin{array}{ccc}
    1 & 2&3 \\
    2 &5&8 \\
    3&8&14
\end{array}\right)$
\end{prim}
$\textbf{(x, y)}=\bm \xi^\text{T}\Gamma\bm \eta=(-1-1\quad 1)\left(\begin{array}{ccc} % неправильная формула + транспонирование
    1 & 2&3 \\
    2 &5&8 \\
    3& 8 &14
\end{array}\right)\left(\begin{array}{r}
      0\\
      1 \\
      3
\end{array}\right)=10$

Для выражения матрицы Грама в новом базисе используется следующая формула:
$$\boxed{\Gamma'=S^\text{T}\Gamma S}$$
\section{Типы базисов}
Пусть задан $\textbf{h}_1,\dots, \textbf{h}_n$~---~ортогональный базис.

Тогда $\textbf{x}=\alpha_1 \textbf{h}_1+\dots+\alpha_n \textbf{h}_n$.

Чему равны коэффициенты $\alpha_1, \alpha_2,\dots, \alpha_n$?

В ортогональных базисах скалярное произведение $(\textbf{h}_i, \textbf{h}_j)=0$ при $i \neq j$.

$\textbf{x}=\alpha_1 \textbf{h}_1+\dots+\alpha_n \textbf{h}_n | \cdot \textbf{h}_1$

$(\textbf{x}, \textbf{h}_1)=\alpha_1 \textbf{|h|}_1^2$, где $\alpha_1 =\dfrac{(\textbf{x, h$_1$})}{|\textbf{h}_1|^2}$

Т.~е. $\alpha_i =\dfrac{(\textbf{x, h$_i$})}{|\textbf{h}_i|^2}$, поэтому

$$\textbf{x} = \dfrac{(\textbf{x, h$_1$})}{|\textbf{h}_1|^2} \textbf{h}_1 + \dots + \dfrac{(\textbf{x, h$_n$})}{|\textbf{h}_n|^2} \textbf{h}_n$$

Вектор равен сумме ортогональных проекций этого вектора на базисные вектора данного базиса.

Выполнено только для ортогонального базиса, иначе произведение $\textbf{(h$_i$, h$_j$)} = 0$ при $i \neq j$ не будет выполнено.

