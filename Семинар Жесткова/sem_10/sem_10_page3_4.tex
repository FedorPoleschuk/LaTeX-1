\section{Евклидовы пространства}
\subsection{Определения}
\begin{definition}
Линейное пространство $\mathcal{E}$ называется евклидовы, если на нем задано скалярное произведение.
\end{definition}
\begin{definition}
Скалярное произведение в вещественном линейном пространстве $L$ ставит в~соответствие число, обозначаемое \textbf{(x, y)}, таким образом, что для любых векторов \textbf{x}, \textbf{y}, \textbf{z} и чисел $\alpha$, $\beta$ выполнены следующие условия:
\renewcommand{\labelenumi}{\theenumi)}
\begin{enumerate}
	\item $\textbf{(x, y)} = \textbf{(y, x)}$.
	\item $\textbf{(x+y, z)} = \textbf{(x, z)}+\textbf{(y, z)}$.
	\item $\textbf{(}\alpha\textbf{x, y)} = \alpha \textbf{(x, y)}$.
	\item $\forall \textbf{x}\neq0 \longmapsto \textbf{(x, x)} > 0$.
\end{enumerate}
\end{definition}
Видно, что <<школьное>> скалярное произведение подходит.
Иначе говоря, \emph{задана положительно определенная квадратичная форма}.
\begin{prim}
Является ли в пространстве многочленов степени $ \leq n $ скалярным произведением
$$
\textbf{(p, q)} = \int\limits_{-1}^{1} p(t)q(t)dt
$$
\end{prim}
Проверим условия, определенные для скалярного произведения:
\renewcommand{\labelenumi}{\theenumi)}
\begin{enumerate}
	\item $p\cdot q=q\cdot p$.
	\item[2, 3)] Определённый интеграл обладает свойствами линейности.
	\item[4)] Если $p(t) \not\equiv 0 $ на отрезке $ [-1, 1] $:
	$$
	\textbf{(p, p)} = \int\limits_{-1}^{1}p^2(t)dt,
	$$
	$ p^2(t) $ --- четная функция. Значение этого интеграла --- площадь подграфика. Т.к. $ p(t) \not\equiv 0 $, эта площадь не отрицательна $ \Rightarrow \textbf{(p, p)}>0 $.
\end{enumerate}
Таким образом, так действительно определено скалярное произведение в пространстве многочленов.
\section{Матрица Грама}
\subsection{Определение}
\begin{definition}
Выберем базис $\textbf{e}\, \{\textbf{e$_1$},\dots,\textbf{e$_n$}\}$. Тогда $\textbf{x}= \bm\xi$, $\textbf{y} = \bm\eta$ --- координаты векторов $ \textbf{x} $, $ \textbf{y} $ в базисе $ \textbf{e} $. Вспомним, что тогда скалярное произведение запишется так:
$$
\textbf{(x, y)} = \bm\xi \Gamma \bm\eta,
$$
где 
$$
\Gamma = 
\begin{pmatrix}
\textbf{(e$_1$, e$_1$)} & \textbf{(e$_1$, e$_2$)} & \hdots & \textbf{(e$_1$, e$_n$)}\\
\textbf{(e$_2$, e$_1$)} & \textbf{(e$_2$, e$_2$)} & \hdots & \textbf{(e$_2$, e$_n$)}\\
\vdots & \vdots&\ddots & \vdots\\
\textbf{(e$_n$, e$_1$)} & \textbf{(e$_n$, e$_2$)} & \hdots & \textbf{(e$_n$, e$_n$)}\\
\end{pmatrix}
$$
--- матрица Грама.
\end{definition}