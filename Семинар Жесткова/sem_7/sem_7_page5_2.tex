\newpage
\begin{prim}
Диагонализировать матрицу:

$$A =
	\begin{pmatrix*}[r]
		0 & -1 & 0\\
		1 & 0  & 0\\
		0 & 0  & 1\\
	\end{pmatrix*}\\
$$	
\end{prim}\\

$
\det~(A - \lambda E) = 0 \Rightarrow % мат функции
    \begin{vmatrix*}[c] % лучше центрирование
        -\lambda & -1 & 0\\
        1 & -\lambda  & 0\\
        0 & 0  & 1-\lambda\\
    \end{vmatrix*}
= (\lambda - 1)(\lambda^{2} + 1) = 0 $
\\
$\lambda = 1, ~ \underbrace {\lambda = \pm ~i}_
		{\substack{
		\text{отвечают за}\\
		\text{инвариантную}\\
		\text{плоскость} }} \\
\\
\\
\begin{pmatrix*}[r]
-1 & -1 & 0 & \vrule & 0\\
1 & -1 & 0 & \vrule & 0\\
0 & 0 & 0 & \vrule & 0\\
\end{pmatrix*};
~L_1 = 
\langle
\left(
\begin{smallmatrix*}[r]
0\\ 0\\ 1\\ 
\end{smallmatrix*}
\right) 
\rangle $
\\
\\
\\
\textbf {Условие диагонализируемости матрицы:} 
\begin{enumerate}
	\item {\itshape В частности}: $A_{n \times n}$ диагонализируема, если $A$ имеет $n$ различных вещественных собственных значений. % перед двоеточием пробелов не ставят
	\item {\itshape В общем случае}: $A$ диагонализируема $\Leftrightarrow$ $L$ раскладывается в прямую сумму собственных подпространств. % тройное тире!, да и вообще оно тут и не нужно
	
\end{enumerate}
