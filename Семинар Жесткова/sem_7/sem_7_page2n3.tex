\begin{prim} %\begin{prim} !!!
Найти инвариантные подпространства в $\mathbb{R}^3$ относительно $\varphi$. %подпрострашнства, R^3!!!!
\end{prim}\\

\begin{wrapfigure}{r}{0.27\linewidth}
	\def\svgwidth{3cm} % если надо изменить размер
	\input{prim2.pdf_tex}
	\caption{К примеру 1}
	\label{prim1}
	\vspace{-1cm}
\end{wrapfigure}

$A = \left( \begin{array}{rr|r} % ВЫРАВНИВАНИЕ
0&-1&0\\
1&0&0\\ 
\hline
0&0&1
\end{array} \right)$

Из вида $A$ видно, что существует два не пересекающихся инвариантных подпространства.\\
$\textbf{o}, \mathbb{R}^3, \langle \textbf{e$_1$}, \textbf{e$_2$} \rangle, \langle \textbf{e$_3$} \rangle.$\\ % Векторы жирным!!!!!  

\subsection{Геометрический смысл матрицы преобразования}
Поговорим о геометрии. Научимся определять по внешнему виду матрицы ее геометрический смысл.
 
$\left( \begin{array}{rrr}
    1 & 0 & 0  \\
    0 & 1& 0\\
    0&0&-1
\end{array}\right)$~---~отражение относительно $\langle \textbf{e$_1$}, \textbf{e$_2$} \rangle.$ %Точки в конце предложений надо ставить!

$\left( \begin{array}{rrr}
    1 & 0 & 0  \\
    0 & 1& 0\\
    0&0&0
\end{array}\right)$~---~проекция.
%Меньше \vspace{}

$\left( \begin{array}{rrr}
    1 & 0 & 0  \\
    0 & 3& 0\\
    0&0&1
\end{array}\right)$~---~растяжение вдоль $\textbf{e$_2$}$ в 3 раза.\\

Нам интересны матрицы вида:

$A=\left( \begin{array}{ccc} %Здесь выравниваение пришлось оставить из-за того, что выравниваение по правому краю происходит по индексам, что выглядит ужасно
    \lambda_1 & 0 & 0  \\
    0 & \lambda_2& 0\\
    0&0&\lambda_3
\end{array}\right)$~---~ обобщённое растяжение $\Leftrightarrow \varphi(\textbf{e$_i$}) = \lambda_i\textbf{e$_i$}$.

\section{Собственный вектор}
\subsection{Определение}
Рассмотрим преобразование  $\varphi$ с матрицей $A$, тогда ненулевой вектор $\textbf{x}$ называется \textsf{собственным вектором}, если  $\varphi(\textbf{x}) = \lambda \textbf{x}$; $\lambda$~---~собственное значение. %Точки в конце предложений!!! + лучше выделять не почеркиванием, а рубленым шрифтом + векторы жирным!!!

Множество собственных векторов, отвечающих одному и тому же собственному значению, образует \textsf{собственное пространство}.
\subsection{Свойства}
\begin{predlog} % Оформление под теорему!
Собственный вектор (и только он) порождает одномерное инвариантное подпространство. % "Собственный вектор (и только он) порождает одномерное инвариантно пространство" - я конечно все понимаю, но "инвариантно пространство" ни в какие ворота, так еще и породить пространство он ну никак не может.
\end{predlog}
\begin{proof}
Рассмотрим инвариантное подпространство  $\langle \textbf{x} \rangle \Rightarrow \varphi(\alpha \textbf{x}) = \alpha \varphi(\textbf{x}) = \alpha\lambda \textbf{x} \in~\langle \textbf{x}\rangle$.
\end{proof}
\section{Алгоритм поиска собственных значений и собственных векторов}
\vspace{-0.5cm}
 $$\varphi(\textbf{x}) = \lambda \textbf{x}$$
 $$\varphi(\textbf{x}) - \lambda \textbf{x} = \textbf{o}$$
Рассмотрим тождественное преобразование $Id$, матрица его преобразования $E$. % Без сокращений!!!
$$(\varphi -\lambda Id)(\textbf{x})=\textbf{o}$$
Перейдём к матричному виду:
\begin{equation}
(\varphi -\lambda E)\textbf{x}=\textbf{o} \tag{$*$}
\end{equation}
Итак, мы получили СЛУ размеров $n\times n$. Она имеет либо одно решение (нулевое), но оно нам не интересно, т.~к. $\textbf{x}\neq\textbf{o}$, либо бесконечно много решений $\Rightarrow A$ должна быть вырожденной $\Rightarrow \det(A-\lambda E) = 0 \rightarrow \lambda_i\text{ --- cобственное значение} \rightarrow (*) \rightarrow \textbf{x}\text{ --- cобственный вектор}.$ %ТОЧКА!, тире ---!!!!
\begin{prim}
Найти собственные значения и собственные векторы.\\
$$A = \left( \begin{array}{rrr}
2 & 2 & 1  \\
-2 & -3& 2\\
3&6&0
\end{array}\right)
$$
\end{prim}\\

Найдём $\lambda$ из условия $\det(A-\lambda E) = 0$:

$\begin{array}{|ccc|}
    2-\lambda & 2 & 1  \\
    -2 & -3 -\lambda& 2\\
     3&6&-\lambda
\end{array} = 0 = -\lambda^3- \lambda^2+17\lambda -15 \Rightarrow \lambda_1 =1, \lambda_2 =3, \lambda_3 = -5$\\