% FIXME надо k-ый шаг прописать
Для построения ортонормированного базиса, каждый вектор нужно разделить на его длину, т.е.
$$
\textbf{e}_i=\cfrac{\textbf{h}_i}{\abs{\textbf{h}_i}}.
$$
\begin{prim}
Ортонормировать систему векторов со стандартным (т.е. $\Gamma = E$) скалярным произведением
\begin{center}
		$
		\textbf{f}_1 = \begin{pmatrixr}
			1&2&1&2
		\end{pmatrixr}^{\text{T}}, \quad \textbf{f}_2 = \begin{pmatrixr}
		4&0&4&1
		\end{pmatrixr}^{\text{T}}, \quad \textbf{f$_3$} = \begin{pmatrixr}
		1&13&-1&-3
		\end{pmatrixr}^{\text{T}}.
		$
\end{center}
~\
\end{prim}
На первом шаге возьмем вектор $\textbf{f}_1$ за основу нового базиса, т.е. $\textbf{h}_1 = \textbf{f}_1 =\begin{pmatrixr}1&2&1&2\end{pmatrixr}^{\text{T}}, \abs{\textbf{h}_1} = \sqrt{10}$. На втором шаге найдем следующий вектор по рекуррентной формуле, полученной выше
$$
\textbf{h}_2 = \textbf{f}_2 - \cfrac{\textbf{(f}_2, \textbf{h}_1\textbf{)}}{\abs{\textbf{h}_1}^2}\, \textbf{h}_1=\begin{psm}
4\\0\\4\\1
\end{psm} - \cfrac{10}{10} \begin{psm}
1\\2\\1\\2
\end{psm}=\begin{psm}
3\\-2\\3\\-1
\end{psm}, \quad \abs{\textbf{h}_2} = \sqrt{23}.
$$
Можно убедиться, что $\textbf{(h}_2, \textbf{h}_1\textbf{)} = 0$.\\
Далее, найдем третий вектор
$$
\textbf{h}_3=\textbf{f}_3-\cfrac{\textbf{(f}_3, \textbf{h}_1\textbf{)}}{\abs{\textbf{h}_1}^2}\, \textbf{h}_1-\cfrac{\textbf{(f}_2, \textbf{h}_2\textbf{)}}{\abs{\textbf{h}_2}^2}\, \textbf{h}_2=\begin{pmatrixr}
2&7&0&-8
\end{pmatrixr}^\text{T}, \quad \abs{\textbf{h}_3} = \sqrt{17}.
$$
Осталось только нормировать полученный базис, т.е. разделить каждый вектор на его длину.\\
Ответ:
$\textbf{e}_1=\frac{1}{\sqrt{10}}\begin{pmatrixr}
	1&2&1&2
\end{pmatrixr}^\text{T},
\textbf{e}_2=\frac{1}{\sqrt{23}}\begin{pmatrixr}
3&-2&3&-1
\end{pmatrixr}^\text{T},
\textbf{e}_3=\frac{1}{\sqrt{17}}\begin{pmatrixr}
2&7&0&8
\end{pmatrixr}^\text{T}.
$

\begin{prim}
	В пространстве многочленов, степени не выше второй, задано скалярное произведение в таком виде:
	$$
	(f, g) = \int\limits_{-1}^1f(t)g(t)dt.
	$$
	Построить ортогональный	базис в этом пространстве.
\end{prim}
За основу возьмем стандартный базис $\underset{\textbf{f}_1\, \textbf{f}_2\, \textbf{f}_3}{\{1, t, t^2\}}$. Пусть первый вектор в нашем новом базисе $\textbf{h}_1 = \textbf{f}_1 = 1$. Найдем длину $\textbf{h}_1$\footnote{Как может показаться длина единицы равна 1. Но т.к. по определению длина вектора --- корень из его скалярного произведения самого на себя, это не так.}:
$$
\abs{\textbf{h}_1}^2=\int\limits_{-1}^11^2dt=2.
$$
Для ортогонализации необходимо найти скалярное произведения $\textbf{f}_1$ и $\textbf{f}_2$. Будем искать их по заданному определению:
$$
\textbf{(f}_1, \textbf{f}_2\textbf{)} = \int\limits_{-1}^1 1\cdot t=0 \Rightarrow 1\perp t.
$$
Теперь подставим числа в рекуррентную формулу и получим второй вектор базиса:
$$
\textbf{h}_2 = \textbf{f}_2 -\cfrac{\textbf{(f}_2, \textbf{h}_1\textbf{)}}{\abs{\textbf{h}_1}^2}\,\textbf{h}_1=\textbf{\textbf{f}}_2\quad \abs{\textbf{h}_2}^2=\int\limits_{-1}^1t^2dt=\cfrac{2}{3}.
$$
Т.к. $\textbf{(f}_3, \textbf{h}_2\textbf{)}=0$,
$$
\textbf{(h}_1, \textbf{f}_3\textbf{)} = \int\limits_{-1}^1 1\cdot t^2=\left.\frac{t^3}{3}\right|_{-1}^1 = \frac{2}{3},
$$
$$
\textbf{h}_3 = \textbf{f}_3 - \cfrac{\textbf{(f}_3, \textbf{h}_2\textbf{)}}{\abs{\textbf{h}_2}^2}\,\textbf{h}_2-\cfrac{\textbf{(f}_3, \textbf{h}_1\textbf{)}}{\abs{\textbf{h}_1}^2}\,\textbf{h}_1,
$$
то
$$
\textbf{h}_3 = t^2-\cfrac{2}{3\cdot2}\cdot1=t^2-\cfrac{1}{3}.
$$
Ответ: $\{1, t, t^2 - \frac{1}{3}\}$.
