\begin{proof}
	Пусть есть пара комплексных корней $\Rightarrow$ существует двумерное инвариантное подпространство $L'$ \textit{без собственных векторов}. Для этого пространства преобразование $\phi$ задается:
	\begin{align}
	\nonumber \phi (L')\!: 
	\left( \begin{array}{ccc}
	\alpha && \beta \\
	\beta && \gamma
	\end{array}\right)
	\text{ , причем для собственных значений~~}
	\begin{vmatrix}
	\alpha - \lambda && \beta \\
	\beta && \gamma - \lambda
	\end{vmatrix} = 0 \Leftrightarrow
	\end{align}
	\hspace*{0.7cm}
	$(\alpha - \lambda)(\gamma - \lambda) - \beta^2 = 0$\\
	\hspace*{0.7cm}
	$\lambda^2-(\alpha+\gamma)\lambda+\alpha\gamma - \beta^2 = 0\\
	\hspace*{0.7cm}
	 D = \alpha^2-2\alpha\gamma + \gamma^2 + 4\beta^2 = (\alpha - \gamma)^2 + 4\beta^2 \geq 0 \\
	\hspace*{0.7cm} \Rightarrow \text{в этом пространстве существует собственный вектор. Противоречие.} $
\end{proof}
\begin{lemma}
	Собственные подпространства самосопряженных преобразований $\phi$ ортогональны (собственные векторы, отвечающие различным собственным значениям, ортогональны).
\end{lemma}
\begin{proof}
	$\left\{
	\begin{matrix}
	\phi(x) = \lambda x\\
	\phi(y) = \mu y \\
	\lambda \neq \mu
	\end{matrix}
	\right. \Rightarrow
	\left\{
	\begin{matrix}
	(\phi(x), y) = \lambda (x, y) \\
	(x, \phi(y)) = \mu (x, y)
	\end{matrix}
	\right. )\ominus \\
	0 = ( \lambda - \mu )(x,y) \Rightarrow
	(x, y) = 0 \Rightarrow x \perp y
	$, что и требовалось доказать
\end{proof}

\section{Центральная теорема}
$\phi$ - самосопряженное $\Leftrightarrow \exists
$ ОНБ из собственных векторов.
\begin{proof}
	($\Rightarrow$) Пусть $U=U_1 \oplus ... \oplus U_n$ - сумма всех собственных подпространств. Докажем, что $U= \mathcal{E} \Leftrightarrow U^\perp = 0$ \\
	$\phi(U^\perp)$ - самосопряженное $	\overset{\text{Лемма 1}}{\Rightarrow}  \exists \lambda \in \mathcal{R} \Rightarrow \exists \text{~с.в.} \in U^\perp \text{~, но все собственные векторы} \in U \Rightarrow U^\perp = 0 \Rightarrow U = \mathcal{E} \Rightarrow$ существует базис из собственных векторов, этот базис можно сделать
\end{proof}