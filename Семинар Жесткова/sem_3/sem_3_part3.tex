Если $A, B, C, D$ --- базис, то $\exists !\ x_1, x_2, x_3, x_4$:
$$
Ax_1+Bx_2+Cx_3+Dx_4=F,
$$
что эквивалентно СЛУ:
$$
\left\{
\begin{array}{rrrrl}
x_1&+2x_2&+x_3&+3x_4&=5\\
-x_1&+5x_2&+x_3&+4x_4&=14\\
x_1&+x_2&&+5x_4&=6\\
x_1&+3x_2&+x_3&+7x_4&=13\\
\end{array}
\right.
$$
Решая эту СЛУ, получим:
$$
\begin{pmatrix}
x_1\\
x_2\\
x_3\\
x_4\\
\end{pmatrix}
=
\begin{pmatrix}
-1\\
2\\
-1\\
1\\
\end{pmatrix}
$$
Т.о., решив систему, убедились в единственности решения (факт базиса).
\begin{prim}
Найти размерность и базис линейной оболочки.
\end{prim}
$$
<(1+t)^3, t^3, t+t^2, 1>
$$
Стандартный базис многочлена: $\left\{1, t, t^2, t^3 \right\}$.
Тогда координаты наших векторов в стандартном базисе есть:
$$
<
\begin{pmatrix}
1\\
3\\
3\\
1\\
\end{pmatrix}
,
\begin{pmatrix}
0\\
0\\
0\\
1\\
\end{pmatrix}
,
\begin{pmatrix}
0\\
1\\
1\\
0\\
\end{pmatrix}
,
\begin{pmatrix}
1\\
0\\
0\\
0\\
\end{pmatrix}
>
$$
Тогда запишем матрицу, аналогично примеру 4:
$$
\begin{pmatrix}
1 & 3 & 3 & 1\\
0 & 0 & 0 & 1\\
0 & 1 & 1 & 0\\
1 & 0 & 0 & 0\\
\end{pmatrix}
$$
1 строка ЛНЗ, ее можно вычеркнуть. Тогда
$$
\dim L'=3, \text{ базис: } \left\{t^3, t^2+t, 1\right\}.
$$
\begin{prim}
Доказать, что
$$
1, t-\alpha, (t-\alpha)^2, \dots, (t-\alpha)^n
$$
--- базис в пространстве многочленов, степени не выше $n$. Найти в этом базисе разложение $P_n(t)$.
\end{prim}
Ответ на эту задачу дал математик Брук Тейлор:
$$
P_n(t)=P_n(\alpha)+\frac{1}{1!}P'(\alpha)(t-\alpha)+\dots+\frac{1}{n!}P^{(n)}_n(\alpha)(t-\alpha)^n
$$
Т.к. данное разложение $\exists !$, это базис. Запишем коэффициенты разложения:
$$
\begin{pmatrix}
P_n(\alpha)\\
\frac{1}{1!}P'_n(\alpha)\\
\hdotsfor{1}\\
\frac{1}{n!}P^{(n)}_n(\alpha)
\end{pmatrix}
$$
\begin{prim}
Найти размерность и базис подпространства, заданного в виде $Ax=0$, где
$$
A = 
\begin{pmatrix}
1 & 2 & 0 & 1\\
3 & 4 & -2 & 5\\
\end{pmatrix}.
$$
\end{prim}
Решения однородной системы образуют линейные подпространства.\\
$
\left(
\begin{array}{cccc|c}
1 & 2 & 0 & 1 & 0\\
3 & 4 & -2 & 5 & 0\\
\end{array}
\right)
\xrightarrow{\text{Алгоритм Гаусса}}
\left(
\begin{array}{cccc|c}
1 & 0 & -2 & 3 & 0\\
0 & 1 & 1 & -1 & 0\\
\end{array}
\right)
$\\
$
\begin{pmatrix}
x_1\\
x_2\\
x_3\\
x_4\\
\end{pmatrix}
=
\begin{pmatrix}
2 & -3\\
-1 & 1\\
1 & 0\\
0 & 1\\
\end{pmatrix}
\begin{pmatrix}
c_1\\
c_2\\
\end{pmatrix}
$\\
Отсюда, получаем линейную оболочку:
$$
<
\begin{pmatrix}
2\\
-1\\
1\\
0\\
\end{pmatrix}
,
\begin{pmatrix}
-3\\
1\\
0\\
1\\
\end{pmatrix}
>
, \qquad \dim L' = 2
$$
Столбцы этой линейной оболочки будут являться базисом в этом линейном подпространстве.
\begin{prim}
Задать подпространство в виде однородной системы
$$
<
\begin{pmatrix}
1\\
2\\
3\\
4\\
\end{pmatrix}
,
\begin{pmatrix}
0\\
1\\
-1\\
1\\
\end{pmatrix}
>.
$$
\end{prim}
Задача по сути является задачей из прошлого семинара:\\
$
\left(
\begin{array}{cc|c}
1 & 0 & x_1\\
2 & 1 & x_2\\
3 & -1 & x_3\\
4 & 1 & x_4\\
\end{array}
\right)
\rightarrow
\left(
\begin{array}{cc|c}
1 & 0 & x_1\\
0 & 1 & x_2-2x_1\\
0 & 0 & -5x_1+x_2+x_3\\
0 & 0 & -2x_2-x_2+x_4\\
\end{array}
\right)
$\\
Для того, чтобы система была совместной, требуется равенство 0 последних двух строк. Отсюда ответ:
$$
\left\{
\begin{array}{rrrrl}
-5x_1&+x_2&+x_3&&=0\\
-2x_1&-x_2&&+x_4&=0\\
\end{array}
\right.
$$
































