\begin{prim} % я использую специальное окружение prim, чтобы не париться с нумерацией
L --- множество n-мерных векторов. % в литературе тире записывается как ---
\end{prim}
a) L' --- множество векторов, координаты которых равны\\
Да, является;  $\dim L'=1$, базис: % использование \dim для функций (в литературе принято писать функции прямыми буквами) 
$\left\{
\begin{pmatrix} % для удобства чтения проще размещать строки друг под другом
1\\ 
1\\ 
\vdots\\
1\\ 
\end{pmatrix}
\right\}$\\
б) L' --- множество векторов,  сумма координат которых равна 0\\
Да, является;  $\dim  L'=n-1$, базис: 
$
\left\{ % Можно использовать \left\{ для того, чтобы скобка подгонялась под нужный размер 
\begin{pmatrix} % окружение smallmatrix почему-то приводит к косякам вроде лишнего пробела перед многоточием. Думаю, проще использовать просто pmatrix, тем более, что место неограничено
1\\ 
0\\ 
\vdots\\ 
0\\ 
-1\\
\end{pmatrix}
, \cdots ,
\begin{pmatrix}
0\\ 
\vdots\\ 
0\\ 
1\\ 
-1\\
\end{pmatrix}
\right\}$\\
в) L' --- множество векторов,  сумма координат которых равна 1\\
Нет, не является.\\

\begin{prim}
L --- множество матриц размера $n \times n$.
\end{prim}
a) L' --- матрицы с нулевой первой строкой\\
Да, является;  $\dim  L'=n^2-n$\\
б) L' --- множество диагональных матриц\\
Да, является;  $\dim  L'=n$\\
в) L' --- множество верхнетреугольных матриц\\
Да, является;  $\dim  L'=\frac{n(n+1)}{2}$   (т.е. $(1+2+\cdots +n)$) \\
г) L' --- множество вырожденных матриц\\
Нет, не является;
$\left( % здесь действительно лучше использовать smallmatrix
	 \begin{smallmatrix}
	 0&0 \\
	 0&1 
	\end{smallmatrix} 
\right)$  
$+$
 $\left(
	 \begin{smallmatrix}
	 1&0 \\
	 0&0 
	\end{smallmatrix} 
\right)$  
$=$
 $\left(
	 \begin{smallmatrix}
	 1&0 \\
	 0&1 
	\end{smallmatrix} 
\right)$  
\\

\begin{prim}
L --- множество функций, определенных на отрезке [0,1].
\end{prim}
a) L' --- множество функций, ограниченных на отрезке [0,1]\\
Да, является. \\
б) L' --- множество строго монотонных функций\\
Нет, не является. \\
в) L' --- множество строго возрастающих функций\\
$0\cdot x = 0 \Longrightarrow$ нет, не является. \\

\section{Примеры и способы задания линейных подпространств}

0 --- тоже линейное пространство
\begin{definition}
Линейная оболочка векторов $a_1, a_2, \cdots , a_k$  $(<a_1, a_2, \cdots , a_k>)$ --- всевозможные линейные комбинации этих векторов:
$$<a_1, a_2, \cdots , a_k>=\left\{ \sum\limits^k _{i= 1} \lambda_i  a_i  , \lambda_i \in R \right\}$$ 
\end{definition}

\begin{prim}
Найти размерность и базис линейной оболочки
\end{prim}
$$
<
\begin{pmatrix}
1\\ 
1\\ 
1\\ 
1\\
\end{pmatrix}
,
\begin{pmatrix}
1\\ 
1\\ 
1\\ 
3\\
\end{pmatrix}
,
\begin{pmatrix}
3\\ 
-5\\ 
7\\ 
2\\
\end{pmatrix}
,
\begin{pmatrix}
1\\ 
-7\\ 
5\\ 
2\\
\end{pmatrix}
>
$$
Т.к. $\dim L'=\Rg A$:
$$
A=
	\begin{pmatrix*}[r]
	 1 & 1 & 1 & 1 \\
	 1 & 1 & 1 & 3 \\
	 3 & -5 & 7 & 2 \\
	 1 & -7 & 5 & 2 \\ 
	\end{pmatrix*} 
\xrightarrow[(3)-3(1)]{ % стрелка позволят писать и под ней
	(2)-(1),\\
	(4)-(1)}
	\begin{pmatrix*}[r]
	 1 & 1 & 1 & 1 \\
	 0 & 0 & 0 & 2 \\
	 0 & -8 & 4 & 1 \\
	 0 & -8 & 4 & 1 \\ 
	\end{pmatrix*} \\
$$
$(3)=(4) \Rightarrow (4)$ вычеркиваем!\\
$\dim  L'=3$, базис: $
\left\{
\begin{pmatrix}
1\\ 
1\\ 
1\\ 
1\\
\end{pmatrix}
,
\begin{pmatrix}
0\\ 
0\\ 
0\\ 
2\\
\end{pmatrix}
,
\begin{pmatrix}
0\\ 
-8\\ 
4\\ 
1\\
\end{pmatrix}
\right\}$

\begin{prim}
(условие --- см. пример 4)
\end{prim}
$$<
	\begin{pmatrix*}[r]
	 6 & 8 & 9 \\
	 0 & 1 & 6 \\
	\end{pmatrix*} 
,
	\begin{pmatrix*}[r]
	 2 & 1 & 1 \\
	 3 & 0 & 1 \\
	\end{pmatrix*} 
,
	\begin{pmatrix*} % из-за выравнивания по правому краю, появляется много лишнего места. имхо, это не очень красиво
	 2 & 6 & 7 \\
	 -6 & 1 & 4 \\
	\end{pmatrix*} 
>$$
Задача не изменится, если взять
$$<
	\begin{pmatrix*}[r]
	 6 \\ 
	 8 \\ 
	 9 \\
	 0 \\ 
	 1 \\ 
	 6 \\
	\end{pmatrix*} 
,
	\begin{pmatrix*}[r]
	 2 \\ 
	 1 \\ 
	 1 \\
	 3 \\ 
	 0 \\ 
	 1 \\
	\end{pmatrix*} 
,
	\begin{pmatrix*}
	 2 \\ 
	 6 \\ 
	 7 \\
	 -6 \\ 
	 1 \\ 
	 4 \\
	\end{pmatrix*} 
> \\
\begin{pmatrix*}
	 6 & 8 & 9 & 0 & 1 & 6 \\
	 2 & 1 & 1 & 3 & 0 & 1 \\
	 2 & 6 & 7 & -6 & 1 & 4 \\
	\end{pmatrix*} 
$$
Т.к. строка (3) ЛНЗ ((1)-2(2)=(3)), ее можно вычеркнуть.\\
$\dim  L'=2$, базис: 
$\left\{
\begin{pmatrix*}[r]
	 6 & 8 & 9 \\
	 0 & 1 & 6 \\
	\end{pmatrix*} 
,
	\begin{pmatrix*}[r]
	 2 & 1 & 1 \\
	 3 & 0 & 1 \\
	\end{pmatrix*} 
\right\}$\\
\begin{prim}
Доказать, что матрицы A, B, C, D образуют базис в пространстве матриц $2 \times 2$ и найти координаты вектора F в этом базисе.
\end{prim}
$$
A=
\begin{pmatrix*}
	 1 & -1 \\
	 1 & 1  \\
	\end{pmatrix*} 
,\  % в мат. режиме для пробелов можно использовать \ 
B=
\begin{pmatrix*}[r]
	 2 & 5 \\
	 1 & 3  \\
	\end{pmatrix*}
,\  
C=
\begin{pmatrix*}[r]
	 1 & 1 \\
	 0 & 1  \\
	\end{pmatrix*} 
, \ 
D=
\begin{pmatrix*}[r]
	 3 & 4 \\
	 5 & 7  \\
	\end{pmatrix*} 
,\  
F=
\begin{pmatrix*}[r]
	 5 & 14 \\
	 6 & 13  \\
	\end{pmatrix*} 
$$
