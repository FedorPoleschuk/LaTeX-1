\section{Определение линейного пространства}
\begin{definition} % definition
Пространство $L$ ~---~ \underline{линейное пространство}, если:
 \begin{itemize}
     \item $\forall x, y \in L: x + y \in L$
     \item $\forall x \in L, \forall \alpha \in \mathbb{R} : \alpha x \in L$
 \end{itemize}

  + 8 аксиом:
  \begin{enumerate}
      \item $x + y = y+x$
      \item $(x+y) +z=x +(y+z)$
      \item $\exists\ o : \forall x \rightarrow x+o=x$
      \item $\exists\ (-x):\forall x \rightarrow  x+(-x) = o$
      \item $\alpha(x+y) = \alpha x+\alpha y$
      \item $(\alpha +\beta)x = \alpha x+\beta x$
      \item $(\alpha\beta) x=\alpha(\beta x)$
      \item $\exists\ 1: x\cdot 1= x$
  \end{enumerate}
\end{definition} 
  \underline{Вектор}~---~ элемент линейного пространства.
  
  \begin{itemize}
      \item Понятия ЛЗ и ЛНЗ со всеми вытекающими свойствами полностью из аналита.
  \end{itemize}
  
\begin{definition}
  \underline{Базис в $L$}~---~ конечная, упорядоченная ЛНЗ система векторов, такая что каждый вектор из $L$ по ней раскладывается.
\end{definition}
  Если базис состоит из $n$ векторов, то пространство называется \underline{$n$-мерным} ($\dim L = n$). % использование \dim для функций (в литературе принято писать функции прямыми буквами)
  \vspace{3mm}
  
   Примеры:
   \begin{itemize}
       \item Векторы в $3^x$ $(\dim L = n)$. Базис: $\left\{\left(\begin{array}{c} % зачем создавать по 2 столба? Можно ограничиться окружением \begin{pmatrix}
            1\\  
            0\\
            0\\
        \end{array}\right),\left(\begin{array}{c}
            0\\  
            1\\
            0\\
        \end{array}\right),\left(\begin{array}{c}
            0\\  
            0\\
            1\\
        \end{array}\right)\right\}$
        \item Столбцы высотой $n$ $(\dim L = n)$. Базис: $\left\{\left(\begin{array}{c}
            1\\  
            0\\
            \vdots\\
            0\\
        \end{array}\right),\ldots,\left(\begin{array}{c}
            0\\
            0\\
            \vdots\\
            1\\
        \end{array}\right)\right\}$
        \item Матрицы $m\times n$ $(\dim L =m\times n)$. Базис: $\left\{\begin{pmatrix}
            1 & 0 & \cdots & 0 \\  
            0 & 0 & \cdots & 0 \\    
            \hdotsfor{4} \\
            0 & 0 & \cdots & 0
            \end{pmatrix},\begin{pmatrix}
            0 & 1 & \cdots & 0 \\  
            0 & 0 & \cdots & 0 \\       
            \hdotsfor{4} \\
            0 & 0 & \cdots & 0
            \end{pmatrix},\dots,\begin{pmatrix}
            0 & 0 & \cdots & 0 \\  
            0 & 0 & \cdots & 0 \\       
            \hdotsfor{4} \\
            0 & 0 & \cdots & 1
            \end{pmatrix}\right\}$
        \item Множество функций, определённых на отрезке $[0,1]$ % избегаем сокращений.
        \item Многочлены $(\dim L = \infty)$
        \item Многочлены степени $\leq n$ $(\dim L = n+1)$.~Базис:$\{1, t, t^2, \dots,  t^n\}$ %после запятых все-таки нужны пробелы
    \end{itemize}
   
\begin{definition}
    \underline{Линейное подпространство}.~$L'$~---~ линейное подпространство в $L$, если:
     \begin{itemize}
     \item $\forall x, y \in L': x + y \in L'$ % пробел между x, y
     \item $\forall x \in L', \forall \alpha \in \mathbb{R} : \alpha x \in L'$
 \end{itemize}
\end{definition}

  Пример: диагональные матрицы в пространстве обычных матриц.
  \section{Примеры}
  В примерах 1~---~3 вопрос следующий: является ли данное множество линейным подпространством в данном пространстве $L$.