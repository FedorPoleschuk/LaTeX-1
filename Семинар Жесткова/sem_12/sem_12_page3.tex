Если $X$~---~ортогональна $\Rightarrow XX^{\text{T}}=E$
$$\left(\begin{array}{rr}
    65 & 15 \\
     15&5
\end{array}\right)=\left(\begin{array}{rr}
    65 & 15 \\
     15&5
\end{array}\right) \Rightarrow \text{Предложение верно} \Rightarrow \varphi \text{ ортогонально}$$

\section{Полярное разложение}

\begin{theorem}
Любое линейное преобразование евклидова пространства $\varphi$ представимо в виде $\varphi=qы$, где $q$~---~ортогональное преобразование, а $s$~---~самосопряженное преобразование.
\end{theorem}
Иначе говоря, любая матрица $A$ раскладывается в произведение
\begin{equation}
A=QS,\tag{\sun}
\label{sun}
\end{equation}
где $Q$~---~ортогональная матрица, $S$~---~симметрическая матрица.

То есть существует такое ортогональное преобразование $P: \, P^{-1}SP=D$~---~диагональная матрица.

$S=PDP^{-1} \Rightarrow \eqref{sun} \Rightarrow A=\underbrace{QP}_{Q_1}D\underbrace{P^{-1}}_{Q_2} \Rightarrow$ \boxed{A=Q_1DQ_2}, где $Q_1,Q_2$~---~ортогональные матрицы, $D$~---~диагональная.
\section{Билинейные функции на евклидовом пространстве}
\begin{definition}
Линейное преобразование $\varphi$ называется присоединенным к билинейной форме $b(\textbf{x}, \textbf{y})$, если $\forall \textbf{x}, \textbf{y} \in \mathcal{E} \rightarrow b(\textbf{x}, \textbf{y})=(\textbf{x}, \varphi(\textbf{y}))$
\end{definition}
Фиксируем базис $\textbf{e}, \varphi: A, \textbf{x}=\bm\xi, \textbf{y}=\bm\eta, \varphi(\textbf{y})=A\bm\eta.$

Пусть $B$~---~матрица билинейной формы.

$$b(\textbf{x}, \textbf{y}) = \bm\xi^T B\bm\eta=\bm\xi^T \Gamma A \bm\eta$$
$$\boxed{B = \Gamma A}\Rightarrow A=\Gamma^{-1}B$$

\begin{prim}
Найти матрицу присоединенного преобразования.

$\Gamma=\left(\begin{array}{rr}
    1 & 1 \\
     1&3
\end{array}\right) \quad k(\textbf{x})=4x_1^2+16x_1x_2+6x_2^2$
\end{prim}
$\exists T_1\dots T_m$ элементарные преобразования строк такие, что

$$T_1\dots T_m \Gamma=E | \cdot \Gamma^{-1}B$$
$$T_1\dots T_mB=A$$
