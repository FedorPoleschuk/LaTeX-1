\textbf{Алгоритм}

$A=H^{-1}K$ --- присоединенное преобразование, $K$ ---  квадратичная форма;\\
$A$ --- самосопряженное $\Rightarrow \: \exists  $ ОНБ из собственных векторов $\Rightarrow \: H =E.  $\\
\begin{prim}
Привести две квадратичные формы к диагональному виду: 
$$
k(\textbf{x})=4x^2_1+16x_1x_2+6x^2_2; h(\textbf{x})=x^2_1+2x_1x_2+3x^2_2; 
$$ 
$h(\textbf{x})$ положительно определена.
\end{prim}
$
\triangle \\
\det(H^{-1}K-\lambda E)=0 \: \Leftrightarrow \: \det(H^{-1}(K-\lambda H))=0\\
\det H^{-1}\det(K-\lambda H)=0 \: \Rightarrow \: \det(K-\lambda H)=0\\
\blacktriangle\\
$
$
\Gamma =H
\begin{pmatrix*}[r]
 1 & 1 \\
 1 & 3 \\
\end{pmatrix*}
; \:\: K=
\begin{pmatrix*}[r]
 4 & 8 \\
 8 & 6 \\
\end{pmatrix*}
; \:\: \det(K-\lambda H)=0
; \:\: \det
\begin{pmatrix*}[r]
 4-\lambda & 8-\lambda \\
 8-\lambda & 6-3\lambda \\
\end{pmatrix*}
=0 \Rightarrow 
\left[
\begin{aligned}
\lambda=-4\\
\lambda=5\\
\end{aligned}
\right. \\
1)\, \lambda=-4 \:\:\: \textbf{h}_1=
\left(
\begin{matrix*}[r]
-3\\ 2\\ 
\end{matrix*}
\right) \\
2)\, \lambda=5 \:\:\: \textbf{h}_2=
\left(
\begin{matrix*}[r]
3\\ 1\\ 
\end{matrix*}
\right) \\
|\textbf{h}_1|^2=\textbf{(h}_1\textbf{, h}_2\textbf{)}=
\begin{pmatrix*}[r]
-3 & 2\\ 
\end{pmatrix*}
\begin{pmatrix*}[r]
 1 & 1 \\
 1 & 3 \\
\end{pmatrix*}
\begin{pmatrix*}[r]
-3\\ 2\\ 
\end{pmatrix*}
=9, \: |\textbf{h}_1|=3, \: |\textbf{h}_2|=3\sqrt{2}\\
\tilde{\textbf{h}}_1=
\left(
\begin{matrix*}[r]
-1\\ 2/3\\ 
\end{matrix*}
\right)
\:\:\:
\tilde{\textbf{h}}_2=\cfrac{1}{\sqrt{2}}
\left(
\begin{matrix*}[r]
1\\ 1/3\\ 
\end{matrix*}
\right)
$\\
В этом базисе:\\
$
 \Gamma=\widetilde H=E 
\:\:\:\:\:\:
\hat h(\textbf{x})=\tilde x^2_1+\tilde x^2_2;\\
A=
\begin{pmatrix*}[r]
 -4 & 0 \\
 0 & 5 \\
\end{pmatrix*}
=K \Rightarrow \:\: \hat k(\textbf{x})=-4\tilde x^2_1+5\tilde x^2_2
$