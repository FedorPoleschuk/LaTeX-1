\textbf{\large{Свойства:}} \\
\hspace*{0.07cm} 1) Ортогональное преобразование инъективно.
\begin{proof}
	Пусть $x \in Ker~\phi \text{, т.е. } \phi(x) = 0 \\
	(\phi(x), \phi(x)) = (x, x) = 0 \Rightarrow x = 0 \Rightarrow $ инъекция
\end{proof}
\hspace*{-0.5cm}
2) Собственные значения ортогонального преобразования равны $\pm 1$
\begin{proof}
	$ \phi (x) = \lambda x, x \neq 0 \\
	(\phi(x), \phi(x)) = \lambda^2(x, x) = (x, x) \Leftrightarrow \lambda^2 = 1
	$
\end{proof}
\hspace*{-0.5cm}
3) Пусть подпространство $ U \subset \mathcal{E} $. Если $U$ инвариантно относительно $\phi$, то $U^\perp$ инвариантно относительно $\phi$

\begin{prim}
	$\phi$ переводит столбцы матрицы $A$ в столбцы $B$. Скалярное произведение стандартное. Ортогонально ли $\phi$?
\end{prim}
\underline{Первый способ}\\
$A = \left(
\begin{array}{ccc}
4 && 2 \\
7 && 1
\end{array}
\right) \rightarrow B = \left(
\begin{array}{ccc}
8 && 2 \\
1 && -1
\end{array}
\right)$ \\  
\\
$
\hspace*{0.2cm} x \left( \begin{array}{ccc}
4 \\ 7
\end{array} \right) \hspace{0.7cm} \phi(x) \left(
\begin{array}{ccc}
8 \\ 1
\end{array}
\right) \hspace{0.7cm} (x, y) = 15 \\ \\
\hspace*{0.2cm} y \left( \begin{array}{ccc}
2 \\ 1
\end{array} \right) \hspace{0.7cm} \phi(y) \left(
\begin{array}{ccc}
2 \\ -1
\end{array}
\right) \hspace{0.4cm} (\phi(x), \phi(y)) = 15 \\ \\ 
\begin{array}{ccc}
(x, x) = 65 && (\phi(x), \phi(x)) = 65 \\
(y, y) = 5 && (\phi(y), \phi(y)) = 5
\end{array} \\
\hspace*{0.2cm} \Rightarrow \phi - \text{ ортогональное}
$ \\ 
Может возникнуть мысль, что достаточно проверить два соотношения из трех, однако этого оказывается недостаточно: \\
$ (x, z) = (x, \alpha x + \beta y) = \alpha \uline{(x, x)} + \beta \uwave{(x, y)} = (\phi(x), \phi(z)) = (\phi(x), \alpha \phi (x) + \beta \phi (y)) \\ = \alpha( \uwave{\phi(x), \phi(x)}) + \beta (\uline{\phi(x), \phi(y)})
$ \\
\textit{Контрпример:} \\
$
\hspace*{0.2cm} x \left( \begin{array}{ccc}
-2 \\ 2
\end{array} \right) \hspace{0.7cm} \phi(x) \left(
\begin{array}{ccc}
3 \\ 1
\end{array}
\right) \hspace{0.7cm} (x, y) = 6 \hspace*{0.5cm} (\phi(x), \phi(y)) = 15 \\ \\
\hspace*{0.2cm} y \left( \begin{array}{ccc}
-2 \\ 1
\end{array} \right) \hspace{0.7cm} \phi(y) \left(
\begin{array}{ccc}
0 \\ 6
\end{array}
\right) \hspace{0.4cm} \textbf{но!} \hspace{0.2cm} (x, x) = 8 \hspace*{0.5cm} (\phi(x), \phi(x)) = 10 \\ \\ 
\begin{array}{ccc}
(x, x) = 65 && (\phi(x), \phi(x)) = 65 \\
(y, y) = 5 && (\phi(y), \phi(y)) = 5
\end{array}
$ \\ \\
Длины не сохраняются $ \Rightarrow $ не ортогонально! \vspace{0.3cm} \\
\underline{Второй способ: } \\
$ X \left(
\begin{array}{ccc}
4 && 2 \\
7 && 1
\end{array}
\right) = \left(
\begin{array}{ccc}
8 && 2 \\
1 && -1
\end{array} \right) \hspace*{0.5cm} \text{Транспонируя с обеих сторон: } \vspace*{0.2cm} \\
\left(
\begin{array}{ccc}
4 && 7 \\
2 && 1
\end{array} \right) X^T = \left(
\begin{array}{ccc}
8 && 1 \\
2 && -1
\end{array} \right) \hspace*{0.3cm} \text{Умножая второе уравнение на первое \textit{слева}, получим} \vspace{0.2cm} \\
\left(
\begin{array}{ccc}
4 && 7 \\
2 && 1
\end{array} \right) X X^T \left(
\begin{array}{ccc}
4 && 2 \\
7 && 1
\end{array} \right) = \left(
\begin{array}{ccc}
8 && 1 \\
2 && -1
\end{array} \right) \left(
\begin{array}{ccc}
8 && 2 \\
1 && -1
\end{array} \right)
$