\begin{prim}
	а) Доказать, что для проектора $\phi^2=\phi$.\\
	б) Доказать, что если $\phi^2=\phi$ ($\phi\neq0$, $\not\equiv\Id$), то $\phi$ --- проектор на образ $\parallel$ ядру.
\end{prim}
а) Т.к. имеем дело с проектором, все пространство раскладывается на прямую сумму двух подпространств:
$$
L=L_1\oplus L_2\Rightarrow \textbf{x}=\textbf{x$_1$}+\textbf{x$_2$}.
$$
Тогда, применив наше преобразование (т.е. проектор), получим:
$$
\phi(x)=p_1(x)=x_1.
$$
Теперь к полученному результату применим преобразование ещё раз. Т.к. вектор $\textbf{x$_1$}$ лежит в $L_1$, то его проекция на $L_1$ и есть сам вектор \textbf{x$_1$}:
$$
\phi^2(x)=\phi(\phi(\textbf{x}))=\phi(\textbf{x$_1$})=\textbf{x$_1$} \Rightarrow \boxed{\phi^2=\phi}.
$$
б) Пусть $L$ --- линейное пространство, $\phi: \phi^2=\phi$. Пусть $\textbf{y}\neq \textbf{o}$, $\textbf{y} \in \im\phi$, $\textbf{y}\in\ker\phi$.
\begin{enumerate}
	\item $\phi(\textbf{y}) = \textbf{o}$ (т.к. $\textbf{y}\in\ker\phi$).
	\item $\exists \textbf{x}\in L: \phi(\textbf{x}) = \textbf{y}$ (т.к. $\textbf{y}\in\im\phi$).
	\item $\phi^2(\textbf{x})=\phi(\phi(\textbf{x}))\overset{\text{п.2)}}=\phi(\textbf{y})\overset{\text{п.1)}}{=}\textbf{o}$, откуда получаем, что \textbf{y}=\textbf{o}. Противоречие. Отсюда же следует, что
	$$
	\ker\phi\cap\im\phi=\{\textbf{o}\}
	$$
\end{enumerate}
Рассмотрим подпространство $L'$:
$$
L'=\ker\phi\oplus\im\phi\subset L.
$$
$$
\dim L'=\dim\ker\phi+\dim\im\phi=\dim L.
$$
Отсюда следует, что наше подпространство $L'$ и есть линейное пространство $L$:
$$
L'\equiv L \Rightarrow L=\ker\phi\oplus\im\phi.
$$
