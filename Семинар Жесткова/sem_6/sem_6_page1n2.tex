Для начала решим небольшой пример по прошлому семинару.
\begin{prim}
$\phi: M_{2\times 2}\rightarrow M_{2\times 1}, \phi(\textbf{x})=\textbf{x}\begin{psm}
	1\\4\\
	\end{psm}$. Найти матрицу линейного преобразования $A$.
\end{prim}\\
Запишем базисы:\\
$M_{2\times 2}: \textbf{e}
\left\{
\begin{psm}
1&0\\
0&0\\
\end{psm}
,
\begin{psm}
0&1\\
0&0\\
\end{psm}
,
\begin{psm}
0&0\\
1&0\\
\end{psm}
,
\begin{psm}
0&0\\
0&1\\
\end{psm}
\right\}\\
M_{2\times1}: \textbf{f}
\left\{
\begin{psm}
1\\0
\end{psm}
,
\begin{psm}
0\\1
\end{psm}
\right\}.
$\\
Для удобства в общем виде найдём, что значит наше преобразование:\\
$$
\phi(\textbf{x})=\begin{pmatrixr}
a&b\\
c&d\\
\end{pmatrixr}
\begin{pmatrixr}
1\\4
\end{pmatrixr}
=
\begin{pmatrixr}
a+4b\\c+4d
\end{pmatrixr}.
$$
Далее <<прогоним>> через преобразование базис \textbf{e}:\\
$
\phi(e_1)=
\begin{psm}
1&0\\
0&0\\
\end{psm}
\begin{psm}
1\\4
\end{psm}
=
\begin{psm}
1\\0
\end{psm}, \quad
\phi(e_2)=\begin{psm}
4\\0
\end{psm}, \quad
\phi(e_3)=\begin{psm}
0\\1
\end{psm}, \quad
\phi(e_4)=\begin{psm}
0\\4
\end{psm}.
$\\
Отсюда, получаем ответ:
$$
A=\begin{pmatrixr}
1&4&0&0\\
0&0&1&4\\
\end{pmatrixr}.
$$
\section{Рассмотрение ядра и образа}
Рассмотрим $\phi: \underset{\dim L=n}{L}\rightarrow \underset{\dim \overline{L}=m}{\overline{L}}$.\\
Ядро: $\ker\phi: \{\textbf{x} \in L: A\textbf{\textbf{x}}=\textbf{o}\}$\\
Очевидно, что ЛНЗ решения такого уравнения формируют ФСР, а ФСР задаёт линейное подпространство. Вспоминая количество столбцов в ФСР, легко получить формулу:
$$
\fbox{$\dim\ker\phi=n-\Rg A$}.
$$
Образ $\im\phi: \{\textbf{y} \in \overline{L}:\exists \textbf{x}\in L: A\textbf{x}=\textbf{y}\}$.\\
Аналогично $\im\phi \in \overline{L}$ формирует линейное подпространство т.к.
\begin{center}
$
A\textbf{x$_1$}+A\textbf{x$_2$}=A(\textbf{x$_1$}+\textbf{x$_2$})$

$A\alpha \textbf{x}=\alpha A\textbf{x}$.
\end{center}