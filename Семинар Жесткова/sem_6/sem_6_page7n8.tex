$$
A'=
\begin{pmatrix*}[r]
 1 & 1\\
 -1 & 2\\
\end{pmatrix*}
\begin{pmatrix*}[r]
 0 & 2\\
 -1 & -3\\
\end{pmatrix*}
\begin{pmatrix*}[r]
 2 & 1\\
 -1 & -1\\
\end{pmatrix*}
=
\begin{pmatrix*}[r]
 -1 & 0\\
 0 & -2\\
\end{pmatrix*}
$$
\section{Линейные функции}
\begin{definition}
\textsf{Функция} $f(x)$ на линейном пространстве $L$ --- правило, которое $\forall x \in L$ ставит в соответствие $f(x) \rightarrow \mathbb{R} $.

Функция $f$ линейная, если 
$$
\left\{
\begin{array}{rl} % некрасивое выравнивание 
f(x+y)&=f(x)+f(y)\\
f(\alpha x)&=\alpha f(x)\\
\end{array}
\right.
$$
\end{definition}


Это частный случай линейного отображения при $m=1$.\\
Примеры:\\ % enumerate, не стал исправлять
а) Присвоить вектору его $i$-тую координату.\\
б) Скалярное произведение \textbf{(\textbf{x}, \textbf{a})}, где \textbf{a} --- фиксированный вектор в $\mathbb{R}^3$.\\
в) Определённый интеграл.\\

$\underset{1 \times n}A$- строка функции $A=(\varphi_1 \cdots \varphi_n)$, где $\varphi_i$ --- образ $i$-го базисного вектора (т.е. $\varphi_i=\varphi$(\textbf{e$_i$}))\\

Линейные функции образуют линейное пространство.

\begin{prim}
Может ли $ \forall x \in L$ выполняться:\\
а) $f(x)>0$? Ответ: нет, так как нет нуля;\\
б) $f(x)\geq 0$? Ответ: только если $f(x)\equiv0$;\\
в) $f(x)=\alpha $? Ответ: только для $\alpha\equiv0$, $f(x)\equiv0$.\\
\end{prim}
\begin{prim}
$P(t)$ ---  многочлен степени $\leq n$, $f(P(t))=P'(1)$. Найти $A$.
\end{prim}\\

Базис: $\{ 1, t,\cdots, t^n\}$\\
$$
\begin{array}{rl}
\varphi (\text{\textbf{e$_1$}})=&0\\
\varphi (\text{\textbf{e$_2$}})=&1\\
\varphi (\text{\textbf{e$_3$}})=&2\\
\hdotsfor{2}\\
\varphi (\text{\textbf{e$_{n+1}$}})=&n\\
\end{array}
$$
Отсюда получаем ответ:
$$A=
\begin{pmatrix*}[r]
 0 & 1 & 2 & \cdots & n\\
\end{pmatrix*}$$