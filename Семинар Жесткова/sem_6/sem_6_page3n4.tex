\\
$\left(\begin{array}{rrrr|r}
    0 & 0 & 2 & -2 & 4   \\
    2 & -4& 1&1&0\\
    -1&2&1&-2&3
\end{array}\right)\xrightarrow[(1)\leftrightarrow(3)]{(3)\cdot0.5;(1)\cdot(-1)} \left( \begin{array}{rrrr|r}
    -1&2&1&-2&3   \\
    2 & -4& 1&1&0\\
    0&0&1&-1&2
\end{array}\right) \xrightarrow{(2)-2(1)} \left( \begin{array}{rrrr|r}
    -1&2&1&-2&3   \\
    0 & 0& 3&-3&6\\
    0&0&1&-1&2
\end{array}\right) 
\rightarrow\\
\xrightarrow{(1)+(2)} 
\begin{matrix}
\overset{
\left(\begin{array}{rrrr|r} % приписывание иксов в матрице!??!?!?!
    1&-2&0&1&-1   \\
    0&0&1&-1&2 \\
\end{array}\right)
}{
\begin{matrix*}
 x_1&x_2&x_3&x_4& \ \ \ \ \ 
\end{matrix*}
}
\end{matrix}
\rightarrow 
\begin{matrix}
\overset{\left(\begin{array}{rrrr|r}
    1&0&-2&1&-1   \\
    0&1&0&-1&2 \\
\end{array}\right)
}{
\begin{matrix*}
x_1&x_3&x_2&x_4& \ \ \ \ \ 
\end{matrix*}
}
\end{matrix}$
\vspace{3mm}

$\left( \begin{array}{r}
    x_1\\
    x_3\\
    x_2\\
    x_4
\end{array}\right) = \left( \begin{array}{r}
    -1\\
    2\\
    0\\
    0
\end{array}\right) + \left( \begin{array}{rr}
    2 & -1\\
    0&1\\
    1&0\\
    0&1
\end{array}\right) \left( \begin{array}{r}
    c_1\\
    c_2
\end{array}\right)$
\vspace{3mm}

$\left( \begin{array}{r}
    x_1\\
    x_2\\
    x_3\\
    x_4
\end{array}\right) = \left( \begin{array}{r}
    -1\\
    0\\
    2\\
    0
\end{array}\right) + \left( \begin{array}{r}
    2\\
    1\\
    0\\
    0
\end{array}\right) c_1 + \left( \begin{array}{r}
    -1\\
    0\\
    1\\
    0
\end{array}\right) c_2 $

\begin{prim}
Найти ядро отображения.
\end{prim}\\

Для этого нужно решить СЛУ $A\textbf{x}=\textbf{o} \Rightarrow$

$\Rightarrow\ker\varphi:\langle \left( \begin{array}{r} % МАТФУНКЦИИ
    2\\
    1\\
    0\\
    0
\end{array}\right), \left( \begin{array}{r}
    -1\\
    0\\
    1\\
    0
\end{array}\right)\rangle$\footnote{В примере 2 как раз и был вектор из $\ker\phi$}, \quad $\dim\ker\phi=2$.

\begin{prim}
Найти образ $\im\varphi$.
\end{prim}\\

$\im\varphi : \langle \left( \begin{array}{r}
    0\\
    2\\
    -1
\end{array}\right),\left( \begin{array}{r}
    0\\
    -4\\
    2
\end{array}\right),\left( \begin{array}{r}
    2\\
    1\\
    1
\end{array}\right),\left( \begin{array}{r}
    -2\\
    1\\
    -2
\end{array}\right) \rangle$

$A = \begin{pmatrixr}
0 & 2 & -1\\
0 & -4 & 2\\
2 & 1 & 1\\
-2&1&-2\\ \end{pmatrixr}$ \\
Очевидно, что $(2) = -2(1), (4) = (3)+(1) \Rightarrow$ 2 и 4 строку можно вычеркнуть. \\
$\im\varphi :\langle \left( \begin{array}{r}
    0\\
    2\\
    -1
\end{array}\right), \left( \begin{array}{r}
    2\\
    1\\
    1
\end{array}\right) \rangle$\\
$\dim \im\varphi = 2$\\
$\dim \im \varphi +\dim \ker \varphi =4$




\section{Два важных частных случая}

\begin{enumerate}
    \item Если \underline{$\dim\ker\varphi=0$}:
    
    $\dim \im \varphi = n = \Rg A \Rightarrow$ (столбцы ЛНЗ)
    
    $\textbf{y} \in \overline{L}$
    $\ker\varphi = {0}$\\
    Пусть $\textbf{y}=A\textbf{x$_1$}=A\textbf{x$_2$}$
    
    $A(\textbf{x$_1$}-\textbf{x$_2$}) = 0 \Rightarrow \textbf{x$_1$} - \textbf{x$_2$} \in \ker\varphi =\{0\} \Rightarrow \textbf{x$_1$}=\textbf{x$_2$}$
    
    Если $\ker\varphi = \{0\}$, то это \underline{инъекция}.
    
    Оказывается, верно и обратное:
    
    Отображение инъективно $\Leftrightarrow$ $\ker\varphi = {0}$
    
    Докажем в другую сторону:
    
    Пусть $\dim\ker \varphi \geq 1  \Rightarrow \exists \textbf{x$_0$} \neq 0 \in \ker \varphi$
    
    $A\textbf{x}=\textbf{y}$
    
$A(\textbf{x}+\textbf{x$_0$}) = A\textbf{x}+A\textbf{x$_0$} = \textbf{y}$~---~противоречие инъекции $\Rightarrow \ker\varphi = \{0\}$

Число прообразов = $0, 1, \infty$
  \item Если \underline{$\im\varphi = \mathbb{R}^m = \overline{L}$}:
  
  $\dim \im\varphi = m = \Rg A$~---~строки ЛНЗ $\leftarrow$ \underline{сюръекция}
\end{enumerate}