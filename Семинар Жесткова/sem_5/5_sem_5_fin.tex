Понимая пример 5 как отображение, можно заметить, что $\dim(\im\varphi)=2$,  а значит, для описания всевозможных результатов в $\bar L$ можно было выбрать базис из двух векторов.\\
$ f= \left\{ \left(
\begin{smallmatrix*}[r]
0\\ 1\\ 0\\
\end{smallmatrix*}
\right) 
, 
\left(
\begin{smallmatrix*}[r]
0\\ 0\\ 1\\
\end{smallmatrix*}
\right) 
\right\}$\\ 
$\varphi (\textbf{e$_1$})=\left(
\begin{smallmatrix*}[r]
-1\\ 2\\
\end{smallmatrix*}
\right) $
,
$\varphi (\textbf{e$_2$})=\left(
\begin{smallmatrix*}[r]
1\\ 0\\
\end{smallmatrix*}
\right) $
,
$\varphi (\textbf{e$_3$})=\left(
\begin{smallmatrix*}[r]
0\\ 1\\
\end{smallmatrix*}
\right) $;\\
$$A=
\begin{pmatrixr}
-1 & 1 & 0\\
 2 & 0 & 1\\
\end{pmatrixr}
$$
\begin{prim}
$ \varphi : \underset{\textbf{e$_1$}, \textbf{e$_2$}, \textbf{e$_3$}}L \Rightarrow \underset{\textbf{f$_1$}, \textbf{f$_2$}, \textbf{f$_3$}, \textbf{f$_4$}}{\bar L} $\\

$L: \langle\underset{=\text{\textbf{$e_1$}}}1, \underset{=\text{\textbf{$e_2$}}}x, \underset{=\text{\textbf{$e_3$}}}{x^2}\rangle$ , 
$L: \langle\underset{=\text{\textbf{$f_1$}}}1, \underset{=\text{\textbf{$f_2$}}}x, \underset{=\text{\textbf{$f_3$}}}{x^2}, \underset{=\text{\textbf{$f_4$}}}{x^3}\rangle$ \\
\end{prim}
$\varphi (x)=\int\limits^x_0 f(t)dt$\\
$$
\left.
\begin{aligned}
\varphi (e_1)=\int\limits^x_0 1 dt=x
\left(
\begin{smallmatrix*}[r]
0\\ 1\\ 0\\ 0\\
\end{smallmatrix*}
\right) \\
\varphi (e_2)=\int\limits^x_0 t dt=\frac{x^2}{2}
\left(
\begin{smallmatrix*}[r]
0\\ 0\\ \frac{1}{2}\\ 0\\
\end{smallmatrix*}
\right) \\
\varphi (e_3)=\int\limits^x_0 t^2 dt=\frac{x^3}{3}
\left(
\begin{smallmatrix*}[r]
0\\ 0\\  0\\ \frac{1}{3}\\
\end{smallmatrix*}
\right) 
\end{aligned}
\right\}
A=
\begin{pmatrix*}[r]
 0 & 0 & 0\\
 1 & 0 & 0\\
 0 & \frac{1}{2} & 0\\
 0 & 0 & \frac{1}{3}\\
\end{pmatrix*}
$$