\documentclass[a4paper,12pt]{article}
%%% Работа с русским языком % для pdfLatex
\usepackage{cmap}					% поиск в~PDF
\usepackage{mathtext} 				% русские буквы в~фомулах
\usepackage[T2A]{fontenc}			% кодировка
\usepackage[utf8]{inputenc}			% кодировка исходного текста
\usepackage[english,russian]{babel}	% локализация и переносы
\usepackage{indentfirst} 			% отступ 1 абзаца
\usepackage{gensymb}				% мат символы?

%%% Работа с русским языком % для XeLatex
%\usepackage[english,russian]{babel}   %% загружает пакет многоязыковой вёрстки
%\usepackage{fontspec}      %% подготавливает загрузку шрифтов Open Type, True Type и др.
%\defaultfontfeatures{Ligatures={TeX},Renderer=Basic}  %% свойства шрифтов по умолчанию
%\setmainfont[Ligatures={TeX,Historic}]{Times New Roman} %% задаёт основной шрифт документа
%\setsansfont{Comic Sans MS}                    %% задаёт шрифт без засечек
%\setmonofont{Courier New}
%\usepackage{indentfirst}
%\frenchspacing

%%% Дополнительная работа с математикой
\usepackage{amsfonts,amssymb,amsthm,mathtools}
\usepackage{amsmath}
\usepackage{icomma} % "Умная" запятая: $0,2$ --- число, $0, 2$ --- перечисление
\usepackage{upgreek}

%% Номера формул
%\mathtoolsset{showonlyrefs=true} % Показывать номера только у тех формул, на которые есть \eqref{} в~тексте.

%%% Страница
\usepackage{extsizes} % Возможность сделать 14-й шрифт

%% Шрифты
\usepackage{euscript}	 % Шрифт Евклид
\usepackage{mathrsfs} % Красивый матшрифт

%% Свои команды
\DeclareMathOperator{\sgn}{\mathop{sgn}} % создание новой конанды \sgn (типо как \sin)
\DeclareMathOperator{\rg}{\mathop{rg}}
\DeclareMathOperator{\Rg}{\mathop{Rg}}
\usepackage{csquotes} % ещё одна штука для цитат
\newcommand{\pd}[2]{\ensuremath{\cfrac{\partial #1}{\partial #2}}} % частная производная
\newcommand{\abs}[1]{\ensuremath{\left|#1\right|}} % модуль
\renewcommand{\phi}{\ensuremath{\varphi}} % греческая фи
\newcommand{\pogk}[1]{\!\left(\cfrac{\sigma_{#1}}{#1}\right)^{\!\!\!2}\!} % для погрешностей

% Ссылки
\usepackage{color} % подключить пакет color
% выбрать цвета
\definecolor{BlueGreen}{RGB}{49,152,255}
\definecolor{Violet}{RGB}{120,80,120}
% назначить цвета при подключении hyperref
\usepackage[unicode, colorlinks, urlcolor=blue, linkcolor=blue, pagecolor=blue, citecolor=blue]{hyperref} %синие ссылки
%\usepackage[unicode, colorlinks, urlcolor=black, linkcolor=black, pagecolor=black, citecolor=black]{hyperref} % для печати (отключить верхний!)


%% Перенос знаков в~формулах (по Львовскому)
\newcommand*{\hm}[1]{#1\nobreak\discretionary{}
	{\hbox{$\mathsurround=0pt #1$}}{}}

%%% Работа с картинками
\usepackage{graphicx}  % Для вставки рисунков
\graphicspath{{images/}{images2/}}  % папки с картинками
\setlength\fboxsep{3pt} % Отступ рамки \fbox{} от рисунка
\setlength\fboxrule{1pt} % Толщина линий рамки \fbox{}
\usepackage{wrapfig} % Обтекание рисунков и таблиц текстом
\usepackage{multicol}

%%% Работа с таблицами
\usepackage{array,tabularx,tabulary,booktabs} % Дополнительная работа с таблицами
\usepackage{longtable}  % Длинные таблицы
\usepackage{multirow} % Слияние строк в~таблице
\usepackage{caption}
\captionsetup{labelsep=period, labelfont=bf}

%%% Оформление
\usepackage{indentfirst} % Красная строка
%\setlength{\parskip}{0.3cm} % отступы между абзацами
%%% Название разделов
\usepackage{titlesec}
\titlelabel{\thetitle.\quad}
\renewcommand{\figurename}{\textbf{Рис.}}		%Чтобы вместо figure под рисунками писал "рис"
\renewcommand{\tablename}{\textbf{Таблица}}		%Чтобы вместо table над таблицами писал Таблица

%%% Теоремы
\theoremstyle{plain} % Это стиль по умолчанию, его можно не переопределять.
\newtheorem{theorem}{Теорема}[section]
\newtheorem{proposition}[theorem]{Утверждение}

\theoremstyle{definition} % "Определение"
\newtheorem{definition}{Определение}[section]
\newtheorem{corollary}{Следствие}[theorem]
\newtheorem{problem}{Задача}[section]

\theoremstyle{remark} % "Примечание"
\newtheorem*{nonum}{Решение}
\newtheorem{zamech}{Замечание}[theorem]

%%% Правильные мат. символы для русского языка
\renewcommand{\epsilon}{\ensuremath{\varepsilon}}
\renewcommand{\phi}{\ensuremath{\varphi}}
\renewcommand{\kappa}{\ensuremath{\varkappa}}
\renewcommand{\le}{\ensuremath{\leqslant}}
\renewcommand{\leq}{\ensuremath{\leqslant}}
\renewcommand{\ge}{\ensuremath{\geqslant}}
\renewcommand{\geq}{\ensuremath{\geqslant}}
\renewcommand{\emptyset}{\varnothing}

%%% Для лекций по инфе
\usepackage{alltt}
\newcounter{infa}[section]
\newcounter{num}
\definecolor{infa}{rgb}{0, 0.2, 0.89}
\definecolor{infa1}{rgb}{0, 0.3, 1}
\definecolor{grey}{rgb}{0.5, 0.5, 0.5}
\newcommand{\tab}{\ \ \ }
\newcommand{\com}[1]{{\color{grey}\##1}}
\newcommand{\num}{\addtocounter{num}{1}\arabic{num}\tab}
\newcommand{\defi}{{\color{infa}def}}
\newcommand{\ini}{{\color{infa}in}}
\newcommand{\rangei}{{\color{infa}range}}
\newcommand{\fori}{{\color{infa}for}}
\newcommand{\ifi}{{\color{infa}if}}
\newcommand{\elsei}{{\color{infa}else}}
\newcommand{\printi}{{\color{infa1}print}}
\newcommand{\maxi}{{\color{infa}max}}
\newcommand{\classi}{{\color{infa}class}}
\newcommand{\returni}{{\color{infa}return}}
\newcommand{\elifi}{{\color{infa}elif}}
\newenvironment{infa}[1]{
	
	\vspace{0.5cm}
	\addtocounter{infa}{1}%
	\noindent{\large \textbf{Программа №\thesection.\arabic{infa}}}\textbf{<<#1>>}%
	\begin{alltt}%
	}{\end{alltt}
	\setcounter{num}{0}
	\vspace{0.1cm}}
%Пример кода:
%\begin{infa}{Поразрядная сортировка}
%	\ \num \defi count_sort(a):\tab \com{определяет нашу функцию}
%	\ \num \tab m = \maxi(a)+1
%	\ \num \tab q = [0]*m
%	\ \num \tab \fori x \ini a:
%	\ \num \tab \tab q[x] += 1
%	\ \num \tab pos = 0
%	\ \num \tab \fori x \ini q:
%	\ \num \tab \tab \fori i \ini \rangei(q[x]):
%	\ \num \tab \tab \tab a[pos] = x
%	\num \tab \tab \tab pos += 1
%\end{infa}

\title{sem_J_1}
\author{Георгий Демьянов}
\date{today}
\usepackage[left=1.27cm,right=1.27cm,top=1.27cm,bottom=2cm]{geometry}

\begin{document}
\paragraph{Семинар 1. Матрицы. Ранг матрицы.}
\section{Про матрицы}
Уже умеем
\begin{itemize}
	\item[-] сложение и умножение на число (поэлементно)
	\item[-] транспонировать
	\item[-] умножать
\end{itemize}

Свойства умножения
\renewcommand{\labelenumi}{\arabic{enumi}.\!\degree}
\begin{enumerate}
	\item $A \cdot B \neq B\cdot A$ (если $A \cdot B = B\cdot A$, то $A,B$~--- перестановочные матрицы)
	\item $A\cdot E= E\cdot A = A$
	\item $(AB)C=A(BC)$
	\item $A(B+C)=AB+AC;\quad (B+C)A=BA+CA$
	\item $\alpha (AB)=(\alpha A)B=A(\alpha B)$
	\item $(AB)^\mathrm{T}=B^\mathrm{T}A^\mathrm{T}$
\end{enumerate}
\textbf{Пример 1}

Верно ли:

а) $(A+B)^2=A^2+2AB+B^2$

Проверка:
$$(A+B)^2=(A+B)(A+B)=A^2+AB+BA+B^2$$
Т.о. в общем случае выражение неверно.

б) $(A+B)^2+(A-B)^2=2(A^2+B^2)$

Проверка:
$$A^2+AB+BA+B^2+A^2-AB-BA+B^2=2(A^2+B^2)$$
Т.о. в общем случае выражение верно.
\section{Элементарные преобразования строк}
\begin{itemize}
	\item[-] умножение строки на число, неравное 0
	\item[-] сложение строк
\end{itemize}
Также, элементарными преобразованиями являются:
\begin{itemize}
	\item[-] добавление к строке другой строки, умноженной на число
	\item[-] перестановка строк
\end{itemize}

\emph{Очевидно, что элементарные преобразования обратимы.}

Рассмотрим: $SA=A'$, где $S$~--- матрица элементарного преобразования.
\begin{itemize}
	\item умножение: $\begin{pmatrix}
	1 & 0 \\ 0 & \lambda
	\end{pmatrix}
	\begin{pmatrix}
	a & b \\ c & d
	\end{pmatrix}=
	\begin{pmatrix}
	a & b \\ \lambda c & \lambda d
	\end{pmatrix}$
	\item сложение: $\begin{pmatrix}
	1 & 0 \\ 1 & 1
	\end{pmatrix}
	\begin{pmatrix}
	a & b \\ c & d
	\end{pmatrix}=
	\begin{pmatrix}
	a & b \\ a + c & b + d
	\end{pmatrix}$
\end{itemize}
Элементарная матрица получается элементарными преобразованиями из единичной.\\
Для преобразования столбцов элементарную матрицу нужно умножать справа.\\
Запись нескольких преобразований: $S_1,...,S_N$, то $S_N \cdot ... \cdot S_1 A$.

Строки $a_1,...,a_k$ матрицы $A$ называются ЛНЗ (линейно-независимыми), если
\begin{itemize}
	\item ЛНЗ: $\alpha_1 a_1+...+\alpha_k a_k = 0 \Leftrightarrow \alpha_1=...=\alpha_k=0,$
	
	называются ЛЗ (линейно-зависимыми), если
	\item ЛЗ: $\exists\ \alpha_1,...,\alpha_k: \alpha_1^2+...+\alpha_k^2 \neq 0 \Rightarrow \alpha_1 a_1+...+\alpha_k a_k = 0$
\end{itemize}

Все свойства из аналита.
\begin{itemize}
	\item Если есть нулевая строка, то матрица ЛЗ
	\item Если часть строк ЛЗ, то и матрица ЛЗ
	\item Любая часть ЛНЗ --- ЛНЗ
\end{itemize}

Квадратная матрица вырожденная, если она содержит ЛЗ строки.

Элементарные преобразования \emph{не нарушают} линейных зависимостей в матрице.

В частности: вырожденная матрица при элементарных преобразованиях перейдёт в вырожденную матрицу, а невырожденная матрица при элементарных преобразованиях перейдёт в невырожденную матрицу.

\begin{theorem}
	Каждая невырожденная матрица с помощью элементарных преобразований может быть превращена в единичную
\end{theorem}
\noindent\textbf{Пример 2}

Привести к $E$.
$$
\begin{pmatrix}
	1 & 2 & 1\\
	1 & 1 & -1\\
	2 & 3 & 1\\
\end{pmatrix}
$$
Решение:
Прямой ход метода Гаусса:
$$
\begin{pmatrix}
1 & 2 & 1\\
1 & 1 & -1\\
2 & 3 & 1\\
\end{pmatrix}
\xrightarrow[(3)-2(1)]{(2)-(1)}
\begin{pmatrix}
1 & 2 & 1\\
0 & -1 & -2\\
0 & -1 & -1\\
\end{pmatrix}
\xrightarrow{(3)-(2)}
\left( \begin{array}{ccc}
1 & 2 & 1\\ \cline{1-1}
0 & \multicolumn{1}{|c}{-1} & -2\\\cline{2-2}
0 & 0 & \multicolumn{1}{|c}{1}\\
\end{array} \right)\ \text{--- ступенчатый вид матрицы}
$$
Обратный ход метода Гаусса:
$$
\xrightarrow[(2)+2(3)]{(1)-(3)}
\begin{pmatrix}
	1 & 2 & 0\\
	0 & -1 & 0\\
	0 & 0 & 1\\
\end{pmatrix}
\xrightarrow{(1)+2(2)}
\begin{pmatrix}
	1 & 0 & 0\\
	0 & -1 & 0\\
	0 & 0 & 1\\
\end{pmatrix}
\xrightarrow{(2)\times (-1)}
\begin{pmatrix}
	1 & 0 & 0\\
	0 & 1 & 0\\
	0 & 0 & 1\\
\end{pmatrix}
$$
\begin{theorem}
	Матрица невырождена $\Leftrightarrow$ раскладывается в произведение элементарных матриц.
\end{theorem}
\begin{proof}
$(\Rightarrow)$: см. метод Гаусса (пример 2).\\
$\exists\  T_1,...,T_M$~--- элементарные преобразования строк: $T_M\cdot ...\cdot T_1 A=E$\\
Элементарные преобразования обратимы $\Rightarrow \exists\  S_1,...,S_M: S_M\cdot ...\cdot S_1 E=A\Leftrightarrow S_M\cdot ...\cdot S_1=A$\\
($\Leftarrow$): $A=S_M\cdot ...\cdot S_1 E$\\
Т.к. единичная матрица невырождена, а элементарные преобразования вырожденности не меняют $\Rightarrow A$ невырождена.
\end{proof}
\section{Обратная матрица}
\begin{definition}
Матрица $X$ обратная к матрице $A$, если
$$XA=AX=E,$$
где $A$~--- невырождена, $X$~--- единственна.
\end{definition}

Свойства:
\begin{enumerate}
	\item $(AB)^{-1}=B^{-1}A^{-1}$
	\item $(A^{\mathrm{T}})^{-1}=(A^{-1})^{\mathrm{T}}$
\end{enumerate}

\textbf{Метод Жордана-Гаусса}\\
$$T_M\cdot...\cdot T_1 A=E \quad|\!\cdot A^{-1}$$
$$T_M\cdot...\cdot T_1 E = A^{-1}$$
\textbf{Пример 3}

Доказать, что $A$ невырождена и найти обратную, если
$$A^2+A+E=O$$
Доказательство:
$$A^2+A+E=O$$
$$A(A+E)+E=O$$
$$A(-A-E)=E$$
$$\det E = 1,\qquad \det(A(-A-E))=\det A\cdot\det (-A-E) \Rightarrow \det A \neq 0.$$
Отсюда же следует, что $$A^{-1}=-A-E$$
\textbf{Пример 4}

$$A^m=O\ \text{--- нильпотентная матрица}$$
Доказать $$(E-A)^{-1}=E+A+A^2+...+A^{m-1}$$
Доказательство:
$$(E-A)^{-1}=E+A+A^2+...+A^{m-1}\quad |\cdot (E-A)$$
$$E=E-A+A-A^2+A^2+...+A^{m-1}-A^m=E$$
\textbf{Пример 5}\vspace{1mm}

а)$\begin{pmatrix}
2 & 5\\
1 & 3\\
\end{pmatrix}
X=
\begin{pmatrix}
2 & 1\\
1 & 1\\
\end{pmatrix}
$
\qquad б)$X
\begin{pmatrix}
2 & 5\\
1 & 3\\
\end{pmatrix}
=
\begin{pmatrix}
2 & 1\\
1 & 1\\
\end{pmatrix}
$\vspace{1mm}\\
a) $AX=B\quad A^{-1}\cdot |$\\
$X=A^{-1}B$\\
б) $XA=B\quad |\cdot A^{-1}$\\
$X=BA^{-1}$\\
Найдём $A^{-1}$:\vspace{2mm}\\
$\left( \begin{array}{cc|cc}
2 & 5 & 1 & 0\\
1 & 3 & 0 & 1\\
\end{array} \right)
\xrightarrow{(1)\times \frac{1}{2}}
\left( \begin{array}{cc|cc}
1 & 5/2 & 1/2 & 0\\
1 & 3 & 0 & 1\\
\end{array} \right)
\xrightarrow{(2)-(1)}
\left( \begin{array}{cc|cc}
1 & 5/2 & 1/2 & 0\\
0 & 1/2 & -1/2 & 1\\
\end{array} \right)
\xrightarrow{(2)\times 2}
\left( \begin{array}{cc|cc}
1 & 5/2 & 1/2 & 0\\
0 & 1 & -1 & 2\\
\end{array} \right)\rightarrow
$\vspace{5mm}
$
\xrightarrow{(1)-5/2\times(2)}
\left( \begin{array}{cc|cc}
1 & 0 & 3 & -5\\
0 & 1 & -1 & 2\\
\end{array} \right)
$\vspace{2mm}

Т.о. $A^{-1}=\left( \begin{array}{cc}
3 & -5\\
-1 & 2\\
\end{array} \right).
$\\
Ответ: a) $X=\left( \begin{array}{cc}
	1 & -2\\
	0 & 1\\
\end{array} \right),
$ б) $X=\left( \begin{array}{cc}
5 & -8\\
2 & -3\\
\end{array} \right).
$
\section{Ранг матрицы}
Пусть у матрицы $A$ $r$~--- ЛНЗ строк и нет ЛНЗ системы строк большего числа. Тогда $r$~--- строчный ранг матрицы.

\begin{definition}
	Строчный ранг матрицы~--- максимальное число ЛНЗ строк.
\end{definition}

\begin{theorem}
	Система из r строк ЛНЗ $\Leftrightarrow \exists$ невырожденная подматрица порядка r. 
\end{theorem}
$$
\left( \begin{array}{cccccc}\cline{3-4}
\cdot & \cdot & \multicolumn{1}{|c}{\cdot} & \cdot & \multicolumn{1}{|c}{\cdot} & \cdot\\
\cdot & \cdot & \multicolumn{1}{|c}{\cdot} & \cdot & \multicolumn{1}{|c}{\cdot} & \cdot\\ \cline{3-4}
\cdot & \cdot & \cdot & \cdot & \cdot & \cdot\\
\cdot & \cdot & \cdot & \cdot & \cdot & \cdot\\
\end{array} \right)
$$
Первые две строки ЛНЗ. Вычерченный фрагмент содержит непропорциональные строки.

\begin{definition}
Подматрица порядка $r$ называется базисной, если она невырождена, а все квадратные подматрицы большего порядка вырождены. 
\end{definition}
\begin{definition}
	Ранг матрицы~--- порядок базисной подматрицы.
\end{definition}
Ранг матрицы равен строчному рангу. Ранг не меняется при элементарных преобразованиях.

Свойства:

$\Rg AB \leq \min(\Rg A, \Rg B)$

\subsection{Алгоритм поиска ранга}
Приводим матрицу к ступенчатому виду. Ранг~--- число ненулевых строк.\\
\textbf{Пример 6}

Найти Rg.\vspace{3mm}\\
$
\left( \begin{array}{ccc}
	1 & 1 & 1\\
	1 & 0 & -1\\
	3 & 2 & 1\\
\end{array} \right)
\xrightarrow[(3)-3(1)]{(2)-(1)}
\left( \begin{array}{ccc}
1 & 1 & 1\\
0 & -1 & -2\\
0 & -1 & -2\\
\end{array} \right)
\xrightarrow{(3)-(2)}
\left( \begin{array}{ccc}
1 & 1 & 1\\ \cline{1-1}
0 & \multicolumn{1}{|c}{-1} & -2\\ \cline{2-3}
0 & 0 & \multicolumn{1}{c|}{0}\\
\end{array} \right) \Rightarrow \Rg =2
$\vspace{3mm}\\
\textbf{Пример 7}

Найти Rg.\vspace{3mm}\\
$
\left( \begin{array}{ccccc}
2 & -1 & 3 & -5 & 1\\
1 & -1 & -5 & 0 & 2\\
3 & -2 & -2 & -5 & 3\\
7 & -5 & -9 & -10 & 8\\
\end{array} \right)
\xrightarrow[(4)\leftrightarrow(1)]{1/5\times(4)}
\left( \begin{array}{ccccc}
1 & -1 & 3 & 2 & 1\\
0 & -1 & -5 & 1 & 2\\
1 & -2 & -2 & 3 & 3\\
2 & -5 & -9 & 7 & 8\\
\end{array} \right)
\xrightarrow[(4)-2(1)]{(3)-(1)}
\left( \begin{array}{ccccc}
1 & -1 & 3 & 2 & 1\\
0 & -1 & -5 & 1 & 2\\
0 & -1 & -5 & 1 & 2\\
0 & -3 & -15 & 3 & 6\\
\end{array} \right)
$\vspace{3mm}\\
3 и 4 строчку можно вычеркнуть, т.к. они ЛЗ. Т.о. $\Rg=2$.\\
\textbf{Пример 8}

Найти Rg в зависимости от параметра.\vspace{3mm}\\
$
\left( \begin{array}{ccccc}
	0 & 0 & 1 & -2 & \alpha\\
	2 & -4 & 3 & -2 & 3\\
	3 & -6 & 2 & 2 & 2\\
	-3 & 6 & 1 & -4 & \beta\\
\end{array} \right)
\xrightarrow{(1)\leftrightarrow(4)}
\left( \begin{array}{ccccc}
2 & -4 & 3 & -2 & 3\\
3 & -6 & 2 & 2 & 2\\
-3 & 6 & 1 & -4 & \beta\\
0 & 0 & 1 & -2 & \alpha\\
\end{array} \right)
\xrightarrow[(3)+3/2\times (1)]{(2)-3/2\times(1)}
\left( \begin{array}{ccccc}
2 & -4 & 3 & -2 & 3\\
0 & 0 & -5/2 & 5 & -5/2\\
0 & 0 & 7/2 & -7 & \beta+9/2\\
0 & 0 & 1 & -2 & \alpha\\
\end{array} \right)\rightarrow\\
\xrightarrow[(4)+2/5\times (1)]{(3)-5/7\times(2)}
\left( \begin{array}{ccccc}
2 & -4 & 3 & -2 & 3\\
0 & 0 & -5/2 & 5 & -5/2\\
0 & 0 & 0 & 0 & \beta+1\\
0 & 0 & 0 & 0 & \alpha-1\\
\end{array} \right)\\
$
Rg = 2 при $\alpha=-\beta=1$.\\
Rg = 3 в остальных случаях.\\
\textbf{Пример 9}

Верно ли $\forall A, B$:\\
а) $\Rg(A+B)=\Rg A+\Rg B$\\
Неверно, например:
$$A=B=E_2$$
б) $\Rg(A+B)\leq \Rg A+\Rg B$\\
Верно. Докажем:\\
$$\Rg(A+B)\leq Rg(\underbrace{A+B}_\text{ЛЗ}|A|B)=\Rg(A|B)$$\\
$$r=\Rg A, \qquad s= \Rg B$$\\
$$\Rg(A|B)\leq r+s$$\\
Т.е.
$$\Rg(A+B)\leq \Rg A+\Rg B.$$

\begin{center}
\vfill \emph{{\small Г. С. Демьянов, gosha-dem@ya.ru}}
\end{center}


\end{document}