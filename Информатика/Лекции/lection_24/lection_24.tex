\documentclass[a4paper,12pt]{article}

%%% Работа с русским языком % для pdfLatex
\usepackage{cmap}					% поиск в~PDF
\usepackage{mathtext} 				% русские буквы в~фомулах
\usepackage[T2A]{fontenc}			% кодировка
\usepackage[utf8]{inputenc}			% кодировка исходного текста
\usepackage[english,russian]{babel}	% локализация и переносы
\usepackage{indentfirst} 			% отступ 1 абзаца

%%% Работа с русским языком % для XeLatex
%\usepackage[english,russian]{babel}   %% загружает пакет многоязыковой вёрстки
%\usepackage{fontspec}      %% подготавливает загрузку шрифтов Open Type, True Type и др.
%\defaultfontfeatures{Ligatures={TeX},Renderer=Basic}  %% свойства шрифтов по умолчанию
%\setmainfont[Ligatures={TeX,Historic}]{Times New Roman} %% задаёт основной шрифт документа
%\setsansfont{Comic Sans MS}                    %% задаёт шрифт без засечек
%\setmonofont{Courier New}
%\usepackage{indentfirst}
%\frenchspacing

%%% Дополнительная работа с математикой
\usepackage{amsfonts,amssymb,amsthm,mathtools}
\usepackage{amsmath}
\usepackage{icomma} % "Умная" запятая: $0,2$ --- число, $0, 2$ --- перечисление
\usepackage{upgreek}

%% Номера формул
%\mathtoolsset{showonlyrefs=true} % Показывать номера только у тех формул, на которые есть \eqref{} в~тексте.

%%% Страница
\usepackage{extsizes} % Возможность сделать 14-й шрифт

%% Шрифты
\usepackage{euscript}	 % Шрифт Евклид
\usepackage{mathrsfs} % Красивый матшрифт

%% Свои команды
\DeclareMathOperator{\sgn}{\mathop{sgn}} % создание новой конанды \sgn (типо как \sin)
\usepackage{csquotes} % ещё одна штука для цитат
\newcommand{\pd}[2]{\ensuremath{\cfrac{\partial #1}{\partial #2}}} % частная производная
\newcommand{\abs}[1]{\ensuremath{\left|#1\right|}} % модуль
\renewcommand{\phi}{\ensuremath{\varphi}} % греческая фи
\newcommand{\pogk}[1]{\!\left(\cfrac{\sigma_{#1}}{#1}\right)^{\!\!\!2}\!} % для погрешностей

% Ссылки
\usepackage{color} % подключить пакет color
% выбрать цвета
\definecolor{BlueGreen}{RGB}{49,152,255}
\definecolor{Violet}{RGB}{120,80,120}
% назначить цвета при подключении hyperref
\usepackage[unicode, colorlinks, urlcolor=blue, linkcolor=blue, pagecolor=blue, citecolor=blue]{hyperref} %синие ссылки
%\usepackage[unicode, colorlinks, urlcolor=black, linkcolor=black, pagecolor=black, citecolor=black]{hyperref} % для печати (отключить верхний!)


%% Перенос знаков в~формулах (по Львовскому)
\newcommand*{\hm}[1]{#1\nobreak\discretionary{}
	{\hbox{$\mathsurround=0pt #1$}}{}}

%%% Работа с картинками
\usepackage{graphicx}  % Для вставки рисунков
\graphicspath{{images/}{images2/}}  % папки с картинками
\setlength\fboxsep{3pt} % Отступ рамки \fbox{} от рисунка
\setlength\fboxrule{1pt} % Толщина линий рамки \fbox{}
\usepackage{wrapfig} % Обтекание рисунков и таблиц текстом
\usepackage{multicol}

%%% Работа с таблицами
\usepackage{array,tabularx,tabulary,booktabs} % Дополнительная работа с таблицами
\usepackage{longtable}  % Длинные таблицы
\usepackage{multirow} % Слияние строк в~таблице
\usepackage{caption}
\captionsetup{labelsep=period, labelfont=bf}

%%% Оформление
\usepackage{indentfirst} % Красная строка
%\setlength{\parskip}{0.3cm} % отступы между абзацами
%%% Название разделов
\usepackage{titlesec}
\titlelabel{\thetitle.\quad}
\renewcommand{\figurename}{\textbf{Рис.}}		%Чтобы вместо figure под рисунками писал "рис"
\renewcommand{\tablename}{\textbf{Таблица}}		%Чтобы вместо table над таблицами писал Таблица
\itemsep=0pt

%%% Теоремы
\theoremstyle{plain} % Это стиль по умолчанию, его можно не переопределять.
\newtheorem{theorem}{Теорема}[section]
\newtheorem{proposition}[theorem]{Утверждение}

\theoremstyle{definition} % "Определение"
\newtheorem{definition}{Определение}[section]
\newtheorem{corollary}{Следствие}[theorem]
\newtheorem{problem}{Задача}[section]

\theoremstyle{remark} % "Примечание"
\newtheorem*{nonum}{Решение}
\newtheorem{zamech}{Замечание}[theorem]

%%% Правильные мат. символы для русского языка
\renewcommand{\epsilon}{\ensuremath{\varepsilon}}
\renewcommand{\phi}{\ensuremath{\varphi}}
\renewcommand{\kappa}{\ensuremath{\varkappa}}
\renewcommand{\le}{\ensuremath{\leqslant}}
\renewcommand{\leq}{\ensuremath{\leqslant}}
\renewcommand{\ge}{\ensuremath{\geqslant}}
\renewcommand{\geq}{\ensuremath{\geqslant}}
\renewcommand{\emptyset}{\varnothing}

%%% Для лекций по инфе
\usepackage{tikz}  
\usetikzlibrary{graphs}
\usepackage{alltt}
\newcounter{infa}[section]
\newcounter{num}
\definecolor{infa}{rgb}{0, 0.2, 0.89}
\definecolor{infa1}{rgb}{0, 0.3, 1}
\definecolor{grey}{rgb}{0.5, 0.5, 0.5}
\newcommand{\tab}{\ \ \ }
\newcommand{\com}[1]{{\color{grey}\# #1}}
\newcommand{\num}{\addtocounter{num}{1}\arabic{num}\tab}
\newcommand{\defi}{{\color{infa}def}}
\newcommand{\globali}{{\color{infa}global}}
\newcommand{\ini}{{\color{infa}in}}
\newcommand{\rangei}{{\color{infa}range}}
\newcommand{\fori}{{\color{infa}for}}
\newcommand{\ifi}{{\color{infa}if}}
\newcommand{\elsei}{{\color{infa}else}}
\newcommand{\printi}{{\color{infa1}print}}
\newcommand{\enumeratei}{{\color{infa1}enumerate}}
\newcommand{\maxi}{{\color{infa}max}}
\newcommand{\classi}{{\color{infa}class}}
\newcommand{\returni}{{\color{infa}return}}
\newcommand{\elifi}{{\color{infa}elif}}
\newcommand{\seti}{{\color{infa}set}}
\newcommand{\noti}{{\color{infa}not}}
\newcommand{\dicti}{{\color{infa}dict}}
\newcommand{\zipi}{{\color{infa}zip}}
\newcommand{\chri}{{\color{infa}chr}}
\newcommand{\ordi}{{\color{infa}ord}}
\newcommand{\leni}{{\color{infa}len}}
\newcommand{\deli}{{\color{infa}del}}
\newcommand{\sortedi}{{\color{infa}sorted}}
\newcommand{\keyi}{{\color{infa}key}}
\newcommand{\lambdai}{{\color{infa}lambda}}
\newcommand{\inti}{{\color{infa}int}}
\newcommand{\inputi}{{\color{infa}input}}
\newcommand{\isi}{{\color{infa}is}}
\newcommand{\Nonei}{{\color{infa}None}}
\newcommand{\whilei}{{\color{infa}while}}
\newcommand{\andi}{{\color{infa}and}}
\newcommand{\fromi}{{\color{infa}from}}
\newcommand{\importi}{{\color{infa}import}}
\newcommand{\continuei}{{\color{infa}continue}}
\newcommand{\mapi}{{\color{infa}map}}
\newcommand{\Falsei}{{\color{infa1}False}}
\newcommand{\listi}{{\color{infa}list}}
\newcommand{\Truei}{{\color{infa1}True}}
\newcommand{\mini}{{\color{infa1}min}}
\newcommand{\breaki}{{\color{infa1}break}}



\newenvironment{infa}[1]{
	
	\vspace{0.5cm}
	\addtocounter{infa}{1}%
	\noindent{\large \textbf{Программа №\thesection.\arabic{infa}.}\ \textbf{#1}}%
	\begin{alltt}%
	}{\end{alltt}
	\setcounter{num}{0}
	\vspace{0.1cm}}
\newenvironment{infanoname}{
	
	%\vspace{0.5cm}
	%\addtocounter{infa}{1}%
	%\noindent{\large \textbf{Программа №\thesection.\arabic{infa}.}\ \textbf{#1}}%
	\begin{alltt}%
	}{\end{alltt}
	\setcounter{num}{0}
	\vspace{0.1cm}}

\usepackage{animate} % Для добавления гифок
\usepackage{xmpmulti}
%Пример кода:
%\begin{infa}{Поразрядная сортировка}
%	\ \num \defi count_sort(a):\tab \com{определяет нашу функцию}
%	\ \num \tab m = \maxi(a)+1
%	\ \num \tab q = [0]*m
%	\ \num \tab \fori x \ini a:
%	\ \num \tab \tab q[x] += 1
%	\ \num \tab pos = 0
%	\ \num \tab \fori x \ini q:
%	\ \num \tab \tab \fori i \ini \rangei(q[x]):
%	\ \num \tab \tab \tab a[pos] = x
%	\num \tab \tab \tab pos += 1
%\end{infa}

\usepackage[left=1.27cm,right=1.27cm,top=1.27cm,bottom=2cm]{geometry}
%\hbox to\textwidth{команда колонтитула}

\usepackage{verbatim}
\begin{document}
\newcounter{lec}
\newcommand{\lec}[1]{\addtocounter{lec}{1} \setcounter{section}{0}%
\begin{center}
{\LARGE ЛЕКЦИЯ \arabic{lec}%
\vspace{2mm}%

\textbf{#1}%
}
\end{center}
}
\newpage
\
\setcounter{lec}{23}
\lec{Кодирование}
\section{Равномерное и неравномерное кодирование}
Есть два глобальных подхода в кодировании текста: равномерное кодирование и неравномерное кодирование.

Обозначим алфавит допустимых символов $\mathbb{A}$.

В неравномерном кодировании код символов разной длины (например, UNICODE UTF 8 --- одна из
самых популярных кодировок).

Рассмотрим четырехбуквенное кодирование. Закодируем буквы <<А>>, <<Б>>, <<В>>, <<Г>> таким образом:
\begin{center}
	$
\begin{array}{rl}
\text{А} &= 0\\
\text{Б} &= 1\\
\text{В} &= 10\\
\text{Г} &= 111\\
\end{array}
$
\end{center}
Тогда запись <<ГАГА>> можно закодировать так:
\begin{center}
ГАГА = 11101110 = БББГА,
\end{center}
т.е. декодирование неоднозначно, такое кодирование плохое.

\textbf{Условие Фано:} \textit{для того, чтобы сообщение, записанное с помощью неравномерного по длине кода, однозначно раскодировалось, достаточно, чтобы никакой код не был началом другого (более длинного) кода.} \textbf{Обратное условие Фано} (\textit{ни один код не является концом (суффиксом) другого}) также является достаточным условием однозначного декодирования неравномерного кода.

Тогда пусть
\begin{center}
$
\begin{array}{rl}
\centering
\text{А} &= 0\\
\text{Б} &= 110\\
\text{В} &= 10\\
\text{Г} &= 111\\
\end{array}
$
\end{center}

Возьмем (для удобства рядом записан столбец в зеркальном отражении):
\begin{center}
$
\begin{array}{rlcr}
\text{A} = &1&\vrule&1\\
\text{Б} = &10&\vrule&01\\
\text{В} = &100&\vrule&001\\
\text{Г} = &000&\vrule&000\\
\end{array}
$
\end{center}
Тогда
\begin{center}
БАГАВА = 10100011001 
\end{center}
--- 1100 нет, т.е в конце ВА. 1000 тоже нет, т.е. по середине ГА. 101 нет, в начале БА. Однозначность есть, хотя и декодировать очень сложно. 

Составим суффиксное дерево. Оно не нужно при декодировании, а при доработке дерева (дополнении алфавита) является полезным инструментом. Способ составления: отзеркаливаем код и строим дерево, в котором каждое ребро --- цифра в коде.

Кодировка в равномерном кодировании UTF-16. Можно закодировать $2^n$ различных символов (мощность алфавита).
%FIXME вставить рисунок префиксного дерева

\section{Поиск подстроки в строке}
\subsection{Наивный поиск подстроки в строке}
\begin{infa}{Примитивный поиск подстроки в строке}
\num s = "abbbbabbaaabbababaabb"
\num subs = "bbbaba"
\num \defi find(s, sub):
\num \tab \fori pos range(0, len(s)-len(sub)+1):
\num \tab \tab \fori i \ini \rangei(\leni(sub)):
\num \tab \tab \tab \ifi sub[i] != s[pos+i]:
\num \tab \tab \tab \tab \breaki
\num \tab \tab \elsei:
\num \tab \tab \tab \returni pos
\num \tab \returni -1
\end{infa}
Построчный комментарий кода:\\
4) Имеет смысл проходить по основной строке, пока входящая строка влезает в рассматриваемый участок.\\
5)--6) Пробегаемся по элементам входящей строки и смотрим, совпадают ли они с элементами основной.\\
7) Если нет, уже можно переходить к следующему элементу основной строки.\\
9) Если прошли по всем элементам строки вхождения, можно выдать ту позицию, начиная с которой есть вхождение.\\
10) Если вхождения нет, выдается ‘-1’.

Сложность алгоритма $O(N\cdot M)$. В итоге алгоритм получается неэффективным.
\subsection{Конечный автомат поиска <<abcd>>}
Смотрим на каждый символ только  по одному разу! Методика хранения автомата: орграф. Если конечный автомат уже построен, то время поиска $O(N)$, $N$ --- длина строки.

Конечный автомат поиска является частным случаем машины Тьюринга, Подход таков:
\begin{enumerate}
	\item Изначально система в фазе ноль.
	\item Сравниваем букву в основной строке с буквой во входящей строке. Если они совпали, то код продвигается на фазу вперед.
	\item Сравниваем следующие буквы. Если они совпали, переходим в фазу два и т.д.
	\item В случае несовпадения фаза становится нулевой.
\end{enumerate}

\begin{infa}{Конечный автомат для поиска подстроки <<abcd>>}
\ \num state = 0
\ \num \fori c in s:
\ \num \tab \ifi state == 0:
\ \num \tab \tab \ifi c == "a":
\ \num \tab \tab \tab state = 1
\ \num \tab \elifi state == 1:
\ \num \tab \tab \ifi c == 'b':
\ \num \tab \tab \tab state = 2
\ \num \tab \tab \elifi c == 'a':
\num \tab \tab \tab state = 1
\num \tab \tab \elsei:
\num \tab \tab \tab state = 0
\num \tab \elifi state == 2:
\num \tab \tab \ifi c == 'c':
\num \tab \tab \tab state = 3
\num \tab \tab \elifi c == 'a':
\num \tab \tab \tab state = 1
\num \tab \tab \elsei:
\num \tab \tab \tab state = 0
\num \tab \elifi state == 3:
\num \tab \tab \ifi c == 'a':
\num \tab \tab \tab state = 1
\num \tab \tab \elifi c == 'd':
\num \tab \tab \tab state = 4
\num \tab \tab \elsei:
\num \tab \tab \tab state = 0
\end{infa}
\section{Расстояние Левенштейна}
\textbf{Расстояние Левенштейна} (также редакционное расстояние или дистанция редактирования) \textbf{между двумя строками в теории информации и компьютерной лингвистике} --- \textit{это минимальное количество операций вставки одного символа, удаления одного символа и замены одного символа на другой, необходимых для превращения одной строки в другую.}

Есть 2 строки Мама и Мим. Мы можем превратить их друг в друга путем вставки символа, удаления символа. Минимальный путь в данном случае --- удаление последнего и замена, т.е. длина пути 2. Так и определяется расстояние Левенштейна.

a[:i], b[:j] --- срезы до i--го и j--го символа.
$F_{ij} = L(a[:i], b[:j])$ --- расстояние Левенштейна. Тогда\\
$$F_{ij}=
\begin{cases}
\text{Последниие буквы совпадают, то } F_{(i-1)(j-1)}\\
1+\min(F_{(i-1)(j-1)}, F_{(i-1)j}, F_{i(j-1)})
\end{cases}
$$






















\begin{center}
	\vfill \emph{{\small Г. С. Демьянов, \href{https://vk.com/id37346992}{VK}\\
С. С. Клявинек, \href{https://vk.com/id85132547}{VK}
}}
\end{center}





\end{document} 