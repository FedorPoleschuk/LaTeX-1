\documentclass[a4paper,12pt]{article}

%%% Работа с русским языком % для pdfLatex
\usepackage{cmap}					% поиск в~PDF
\usepackage{mathtext} 				% русские буквы в~фомулах
\usepackage[T2A]{fontenc}			% кодировка
\usepackage[utf8]{inputenc}			% кодировка исходного текста
\usepackage[english,russian]{babel}	% локализация и переносы
\usepackage{indentfirst} 			% отступ 1 абзаца

%%% Работа с русским языком % для XeLatex
%\usepackage[english,russian]{babel}   %% загружает пакет многоязыковой вёрстки
%\usepackage{fontspec}      %% подготавливает загрузку шрифтов Open Type, True Type и др.
%\defaultfontfeatures{Ligatures={TeX},Renderer=Basic}  %% свойства шрифтов по умолчанию
%\setmainfont[Ligatures={TeX,Historic}]{Times New Roman} %% задаёт основной шрифт документа
%\setsansfont{Comic Sans MS}                    %% задаёт шрифт без засечек
%\setmonofont{Courier New}
%\usepackage{indentfirst}
%\frenchspacing

%%% Дополнительная работа с математикой
\usepackage{amsfonts,amssymb,amsthm,mathtools}
\usepackage{amsmath}
\usepackage{icomma} % "Умная" запятая: $0,2$ --- число, $0, 2$ --- перечисление
\usepackage{upgreek}

%% Номера формул
%\mathtoolsset{showonlyrefs=true} % Показывать номера только у тех формул, на которые есть \eqref{} в~тексте.

%%% Страница
\usepackage{extsizes} % Возможность сделать 14-й шрифт

%% Шрифты
\usepackage{euscript}	 % Шрифт Евклид
\usepackage{mathrsfs} % Красивый матшрифт

%% Свои команды
\DeclareMathOperator{\sgn}{\mathop{sgn}} % создание новой конанды \sgn (типо как \sin)
\usepackage{csquotes} % ещё одна штука для цитат
\newcommand{\pd}[2]{\ensuremath{\cfrac{\partial #1}{\partial #2}}} % частная производная
\newcommand{\abs}[1]{\ensuremath{\left|#1\right|}} % модуль
\renewcommand{\phi}{\ensuremath{\varphi}} % греческая фи
\newcommand{\pogk}[1]{\!\left(\cfrac{\sigma_{#1}}{#1}\right)^{\!\!\!2}\!} % для погрешностей

% Ссылки
\usepackage{color} % подключить пакет color
% выбрать цвета
\definecolor{BlueGreen}{RGB}{49,152,255}
\definecolor{Violet}{RGB}{120,80,120}
% назначить цвета при подключении hyperref
\usepackage[unicode, colorlinks, urlcolor=blue, linkcolor=blue, pagecolor=blue, citecolor=blue]{hyperref} %синие ссылки
%\usepackage[unicode, colorlinks, urlcolor=black, linkcolor=black, pagecolor=black, citecolor=black]{hyperref} % для печати (отключить верхний!)


%% Перенос знаков в~формулах (по Львовскому)
\newcommand*{\hm}[1]{#1\nobreak\discretionary{}
	{\hbox{$\mathsurround=0pt #1$}}{}}

%%% Работа с картинками
\usepackage{graphicx}  % Для вставки рисунков
\graphicspath{{images/}{images2/}}  % папки с картинками
\setlength\fboxsep{3pt} % Отступ рамки \fbox{} от рисунка
\setlength\fboxrule{1pt} % Толщина линий рамки \fbox{}
\usepackage{wrapfig} % Обтекание рисунков и таблиц текстом
\usepackage{multicol}

%%% Работа с таблицами
\usepackage{array,tabularx,tabulary,booktabs} % Дополнительная работа с таблицами
\usepackage{longtable}  % Длинные таблицы
\usepackage{multirow} % Слияние строк в~таблице
\usepackage{caption}
\captionsetup{labelsep=period, labelfont=bf}

%%% Оформление
\usepackage{indentfirst} % Красная строка
%\setlength{\parskip}{0.3cm} % отступы между абзацами
%%% Название разделов
\usepackage{titlesec}
\titlelabel{\thetitle.\quad}
\renewcommand{\figurename}{\textbf{Рис.}}		%Чтобы вместо figure под рисунками писал "рис"
\renewcommand{\tablename}{\textbf{Таблица}}		%Чтобы вместо table над таблицами писал Таблица

%%% Теоремы
\theoremstyle{plain} % Это стиль по умолчанию, его можно не переопределять.
\newtheorem{theorem}{Теорема}[section]
\newtheorem{proposition}[theorem]{Утверждение}

\theoremstyle{definition} % "Определение"
\newtheorem{definition}{Определение}[section]
\newtheorem{corollary}{Следствие}[theorem]
\newtheorem{problem}{Задача}[section]

\theoremstyle{remark} % "Примечание"
\newtheorem*{nonum}{Решение}
\newtheorem{zamech}{Замечание}[theorem]

%%% Правильные мат. символы для русского языка
\renewcommand{\epsilon}{\ensuremath{\varepsilon}}
\renewcommand{\phi}{\ensuremath{\varphi}}
\renewcommand{\kappa}{\ensuremath{\varkappa}}
\renewcommand{\le}{\ensuremath{\leqslant}}
\renewcommand{\leq}{\ensuremath{\leqslant}}
\renewcommand{\ge}{\ensuremath{\geqslant}}
\renewcommand{\geq}{\ensuremath{\geqslant}}
\renewcommand{\emptyset}{\varnothing}

%%% Для лекций по инфе
\usepackage{tikz}  
\usetikzlibrary{graphs}
\usepackage{alltt}
\newcounter{infa}[section]
\newcounter{num}
\definecolor{infa}{rgb}{0, 0.2, 0.89}
\definecolor{infa1}{rgb}{0, 0.3, 1}
\definecolor{grey}{rgb}{0.5, 0.5, 0.5}
\newcommand{\tab}{\ \ \ }
\newcommand{\com}[1]{{\color{grey}\# #1}}
\newcommand{\num}{\addtocounter{num}{1}\arabic{num}\tab}
\newcommand{\defi}{{\color{infa}def}}
\newcommand{\globali}{{\color{infa}global}}
\newcommand{\ini}{{\color{infa}in}}
\newcommand{\rangei}{{\color{infa}range}}
\newcommand{\fori}{{\color{infa}for}}
\newcommand{\ifi}{{\color{infa}if}}
\newcommand{\elsei}{{\color{infa}else}}
\newcommand{\printi}{{\color{infa1}print}}
\newcommand{\enumeratei}{{\color{infa1}enumerate}}
\newcommand{\maxi}{{\color{infa}max}}
\newcommand{\classi}{{\color{infa}class}}
\newcommand{\returni}{{\color{infa}return}}
\newcommand{\elifi}{{\color{infa}elif}}
\newcommand{\seti}{{\color{infa}set}}
\newcommand{\noti}{{\color{infa}not}}
\newcommand{\dicti}{{\color{infa}dict}}
\newcommand{\zipi}{{\color{infa}zip}}
\newcommand{\chri}{{\color{infa}chr}}
\newcommand{\ordi}{{\color{infa}ord}}
\newcommand{\leni}{{\color{infa}len}}
\newcommand{\deli}{{\color{infa}del}}
\newcommand{\sortedi}{{\color{infa}sorted}}
\newcommand{\keyi}{{\color{infa}key}}
\newcommand{\lambdai}{{\color{infa}lambda}}
\newcommand{\inti}{{\color{infa}int}}
\newcommand{\inputi}{{\color{infa}input}}
\newcommand{\isi}{{\color{infa}is}}
\newcommand{\Nonei}{{\color{infa}None}}
\newcommand{\whilei}{{\color{infa}while}}
\newcommand{\andi}{{\color{infa}and}}
\newcommand{\fromi}{{\color{infa}from}}
\newcommand{\importi}{{\color{infa}import}}
\newcommand{\continuei}{{\color{infa}continue}}
\newcommand{\mapi}{{\color{infa}map}}
\newcommand{\Falsei}{{\color{infa1}False}}
\newcommand{\listi}{{\color{infa}list}}
\newcommand{\Truei}{{\color{infa1}True}}
\newcommand{\mini}{{\color{infa1}min}}


\newenvironment{infa}[1]{
	
	\vspace{0.5cm}
	\addtocounter{infa}{1}%
	\noindent{\large \textbf{Программа №\thesection.\arabic{infa}.}\ \textbf{#1}}%
	\begin{alltt}%
	}{\end{alltt}
	\setcounter{num}{0}
	\vspace{0.1cm}}
\newenvironment{infanoname}{
	
	%\vspace{0.5cm}
	%\addtocounter{infa}{1}%
	%\noindent{\large \textbf{Программа №\thesection.\arabic{infa}.}\ \textbf{#1}}%
	\begin{alltt}%
	}{\end{alltt}
	\setcounter{num}{0}
	\vspace{0.1cm}}

\usepackage{animate} % Для добавления гифок
\usepackage{xmpmulti}
%Пример кода:
%\begin{infa}{Поразрядная сортировка}
%	\ \num \defi count_sort(a):\tab \com{определяет нашу функцию}
%	\ \num \tab m = \maxi(a)+1
%	\ \num \tab q = [0]*m
%	\ \num \tab \fori x \ini a:
%	\ \num \tab \tab q[x] += 1
%	\ \num \tab pos = 0
%	\ \num \tab \fori x \ini q:
%	\ \num \tab \tab \fori i \ini \rangei(q[x]):
%	\ \num \tab \tab \tab a[pos] = x
%	\num \tab \tab \tab pos += 1
%\end{infa}

\usepackage[left=1.27cm,right=1.27cm,top=1.27cm,bottom=2cm]{geometry}
%\hbox to\textwidth{команда колонтитула}
\begin{document}
\newcounter{lec}
\newcommand{\lec}[1]{\addtocounter{lec}{1} \setcounter{section}{0}%
\begin{center}
{\LARGE ЛЕКЦИЯ \arabic{lec}%
\vspace{2mm}%

\textbf{#1}%
}
\end{center}
}
\newpage
\
\setcounter{lec}{21}
\lec{Двоичное дерево}
\section{Двоичное дерево. Класс  дерево.}
\textsf{k-ричное дерево} --- дерево, в котором количество дочерних вершин у каждой вершины не больше k штук. При этом троичное дерево является одновременно и четверичным --- просто четвертого ребра еще нет. Последовательность дочерних вершин может быть неупорядоченной. Нам же интересен случай, когда k-ричное дерево является упорядоченным. 

Рассмотрим двоичное дерево, упорядоченное, в общем случае не сбалансированное (в отличии от кучи). Двоичное дерево можно нарисовать так, что мы будем называть дочерние вершины "левая"\ или "правая" (по аналогии с кучей). 

\textsf{Поддерево} (правое/левое) --- подграф дерева, корень которого является дочерней вершиной (правой/левой). 

В прошлом семестре вводился такой способ хранения данных как односвязный список. Оказывается, можно реализовать такое звено, которое подходит для двоичного дерева поиска. Сделаем 2 указателя: на левое поддерево и правое поддерево, также указатель наверх (т.е. на родителя). Из таких звеньев можно собирать дерево.

Удобно считать пустой граф пустым деревом (хотя по определению дерева это неверно).

\begin{infa}{Класс Дерево}
\ \num \classi Node:
\ \num \tab \defi __init__(self, key, value): \com{Создаем звено ключ, значение}
\ \num \tab \tab self.key = key 
\ \num \tab \tab self.value = value 
\ \num \tab \tab self.parent = \Nonei
\ \num \tab \tab self.left = \Nonei
\ \num \tab \tab self.right = \Nonei
\ \num
\ \num \classi Tree:
\num \tab \defi __init__(self):
\num \tab \tab self.root = \Nonei
\num \tab \defi print(self, node):
\num \tab \tab \ifi node \isi \Nonei: \com{Благодаря этому нам не важно, пустое наше поддерево\\\phantom{ 5\ \ \ \ } или нет - крайний случай проверен}
\num \tab \tab \tab \returni
\num \tab \tab self.print(node.left)
\num \tab \tab \printi((node.key, node.value)) \com{Это не совсем обратный ход рекурсии}
\num \tab \tab self.print(node.right)
\end{infa}

\section{Двоичное дерево поиска}
\textsf{Двоичное дерево поиска} --- это корневое двоичное упорядоченное дерево, построенное по следующему правилу для любого звена node: все ключи левого поддерева $key_i<key_{node}$, все ключи правого поддерева $key_j>key_{node}$.

Возьмем такие числа:
$$
6\ 3\ 5\ 4\ 2\ 9\ 7\ 8\ 1\ 11\ 10\ 12\ 0
$$
и изобразим для них двоичное дерево поиска (рис. \ref{fig_dvderevo}).

\begin{figure}
	\centering
	%\vspace{-2.5cm}
	\def\svgwidth{18cm} % если надо изменить размер
	\input{graph_derevo.pdf_tex}
	\caption{Двоичное дерево поиска}
	\label{fig_dvderevo}
	%\vspace{-9cm}
\end{figure}

Алгоритм построения: сначала дерево пустое и ни одного звена нет. Берем числа поочереди. 3 меньше 6, поэтому 6 становится главной. 3 становится левым поддеревом шестерки, т.к. $3<6$ (добавляем числа меньшие корня в левое поддерево, а числа большие корня в правое поддерево). Далее число 5 меньше 5, но больше 3. Поэтому 5 находится в левом поддереве 6, но в правом поддереве тройки. Дальнейшее построение аналогично. Красота такого метода в том, что если <<спроектировать>> числа на прямую, получается числовая ось, на которой числа расставлены в порядке возрастания.

Двоичное дерево поиска работает как бинарный поиск. Количество операций сравнения равно высоте дерева $O(\log_2N)$ (если дерево сбалансировано). Чем оно лучше бинарного поиска в списке? Для добавления элемента требуется то же время ($O(\log_2N)$). Но в списке после того, как мы нашли, куда вставить элемент, требуется сделать циклический сдвиг.

Если у элемента нет дочерних вершин, а его надо удалить, то это сделать просто. Но вот если у него есть одна дочерняя вершина, мы присоединяем оставшееся дерево к верхнему родителю (можно привести аналогию с подчиненными: если начальника подчиненных уволили, то эти подчиненные становятся подчиненными начальника рангом выше).

В случае когда нужно присоеденить 2 дерева (т.е. если мы удалили вершину, у которого было 2 поддерева, например третью) переходим к левому поддереву удаляемой вершины и к самому правому из него добавляем правое поддерево той вершины, которую мы удалили. Минусы: при удалении корневого элемента длина дерева удваивается.

Но есть более оптимальный вариант: после удаления вершины (например тройки) возьмем самый правый элемент левого поддерева этой вершины (т.е. тройки) и поставим его вместо элемента, которого мы удалили (т.е. вместо стройки), что предотвратит от удваивания длины всего дерева (т.е. если бы у двойки было бы правое поддерево, мы бы нашли самый большой элемент в этом поддереве и поставили бы его на место тройки).

Почему не воспользоваться хеш--таблицей? Хеш--таблица не упорядочена, в отличии от двоичного дерева поиска.













%\begin{center}
%	\vfill \emph{{\small Г. С. Демьянов, \href{https://vk.com/id37346992}{VK}\\
%С. С. Клявинек, \href{https://vk.com/id85132547}{VK}
%}}
%\end{center}






\end{document} 